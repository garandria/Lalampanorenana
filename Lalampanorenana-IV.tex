\documentclass[12pt]{article}
\usepackage[bitstream-charter]{mathdesign}
\usepackage[english]{babel}
\usepackage[utf8]{inputenc}
\usepackage[T1]{fontenc}
\usepackage{hyperref}
\usepackage[nameinlink]{cleveref}
\usepackage{quoting}
\usepackage{enumerate}
\usepackage{lipsum}
\usepackage{textcase}
\usepackage{sectsty}
\usepackage{titlecaps}
\usepackage{titlesec}

\title{{\Huge MADAGASIKARA}\\\textsc{Lalam-panorenana ho an'ny\\Repoblika fahefatra}}
\date{Desambra 2010}

\renewcommand\thesection{\Roman{section}}
\renewcommand\thesubsection{\Roman{subsection}}
\renewcommand\thesubsubsection{\Roman{subsection}}
\titleformat{\section}[display]
{\normalfont\Large\bfseries\filcenter}{LOHATENY~\thesection}{0pt}{\MakeUppercase}
\titleformat{\subsection}[display]
{\normalfont\large\bfseries\filcenter}{ZANA-DOHATENY~\thesubsection}{0pt}{\MakeUppercase}
\titleformat{\subsubsection}[display]
  {\normalfont\normalsize\bfseries\filcenter}{TOKO~\thesubsubsection}{0pt}{}

\newcounter{laharana}
\setcounter{laharana}{0}
\newcommand{\andininy}[0]{
  \paragraph{%
    \NoCaseChange{%
      Andininy~\addtocounter{laharana}{1}\thelaharana.}\label{and:\thelaharana}~%
  }%
}

\sloppy

\begin{document}
\maketitle{}

\section*{Savaranonando}
\label{sec:savaranonando}

Ny Vahoaka malagasy masi-mandidy,\\

\noindent
Mametraka ny finoany an'Andriamanitra Andriananahary,\\

\noindent
Hentitra amin'ny fanapahan-keviny hampandroso sy hampiroborobo ny fiaraha-monina
mivelona ao anatin'ny firindrana sy fanajana ny hafa, ny harena mbamin'ny
havitrihana ifotoran'ireo soatoavina ara-kolontsaina sy ara-panahy miainga avy
amin'ny « fanahy maha-olona »,\\

\noindent
Resy lahatra amin'ny tokony hiverenan'ny fiaraha-monina malagasy amin'izay
nihandohany, ny maha-izy azy sy ny maha-malagasy, ary hidirana amin'ny
rafi-baovaon'izao taona arivo izao hitehirizina ny soatoavina sy foto-kevitra
nentim-paharazana mifototra amin'ny fanahy malagasy ahitana « ny fitiavana, ny
fihavanana, ny fifanajàna, ny fitandroana ny aina », ary manome hasina ny
fiainam-piaraha-monina, tsy misy fanavakavahana ara-paritra ara-piaviana,
ara-poko, ara-pinoana, ara-pirehan-kevitra politika, ary tsy anavahana ny lahy
sy ny vavy.\\

\noindent
Mahatsapa fa tena zava-dehibe ny fametrahana ny fomba izorana amin'ny
fampihavanam-pirenena,\\

\noindent
Resy lahatra fa ny Fokonolona, nalamina ho Fokontany, no sehatra iainana,
ivelarana, ifanakalozana sy ifampierana amin'ny fandraisan'anjara mavitriky ny
olom-pirenena,\\

\noindent
Mahatsapa ny maha zava-dehibe tsy manam-paharoa ireo loharanon-karena avy
amin'ny zava-boahary manan'aina sy ireo harena an-kibon'ny tany samihafa
ananan'i Madagasikara ka ilaina arovana ho an'ny taranaka ho avy,\\

\noindent
Mahatsapa fa ny tsy fanajàna ny Lalàmpanorenana na ny fanitsiana azy mba
hanamafisana ny fahefan'ny mpitondra nefa tsy mitondra vokatsoa ho an'ny vahoaka
no antony mahatonga ny krizy miverimberina,\\

\noindent
Noho ny toerana misy ara-politika an'i Madagasikara sy ny fandraisany anjara
an-tsitrapo eo amin'ny fifarimbonan'ny firenen-tsamihafa, ka andraisany ho toy
ny nataony, indrindra~:
\begin{itemize}
\item ny Sata iraisam-pirenena momba ny zon'olombelona~;
\item ny Fifanarahana mikasika ny zon'ny ankizy, ny zon'ny vehivavy, ny
  fiarovana ny tontolo iainana, ny zo ara-tsosialy, ara-toekarena, ara-politika,
  isam-batan'olona ary ara-kolontsaina~;
\end{itemize}

\noindent
Mihevitra fa ny fivelaran'ny maha-izy azy sy ny fiavahan'ny Malagasy rehetra no
fanoitra ilaina ho an'ny fampandrosoana maharitra sy mirindra, ka ny fepetra
arahina indrindra dia:
\begin{itemize}
\item ny fitandroana ny fandriampahalemana, ny firaisan-kina ary ny adidy
  amin'ny fitandroana ny firaisam-pirenena eo amin'ny asa fametrahana politika
  fampandrosoana mifandanja sy mirindra~;

\item ny fanajana sy ny fiarovana ny fahalalahana sy zo fototra~;

\item ny fananganana Fanjakana tan-dalàna izay hahatonga ny mpitondra sy ny
  entina samy ho voafehin'ny fitsipi-dalàna mitovy, eo ambany fanaraha-mason'ny
  Fitsarana mahaleotena~;

\item ny famongorana ny tsy fahamarinana, ny kolikoly, ny tsy fitoviana sy ny
  fanavakavahana amin'ny endriny rehetra;

\item ny fitantanana araka ny tokony ho izy sy ara-drariny ny harena voajanahary
  ilaina amin'ny fampandrosoana ny olombelona~;

\item ny fitantanana tsara eo amin'ny fitondrana ny raharaham-panjakana, amin'ny
  fomba mangarahara eo amin'ny fitantanana sy ny fampandraisan'andraikitra ny
  mpitantana ny fahefam-panjakana~;

\item ny fisarahana sy fifandanjan'ny fahefana ampiharina araka ny fomba
  demokratika~;

\item ny fametrahana ny fitsinjaram-pahefana tena izy, amin'ny alàlan'ny
  fanomezana fizakan-tena malalaka ny vondrombahoaka itsinjaram-pahefana na eo
  amin'ny sehatra iandraiketany na eo amin'ny ara-bola~;

\item ny fitandroana ny fiarovana ny ain'ny olona.
\end{itemize}

\begin{center}
  Dia manambara~:
\end{center}

\section{Ny amin'ny feni-kevitra fototra}
\label{sec:ny-aminny-feni}

\andininy{} Firenena iray ny Vahoaka Malagasy ka mivondrona ho Fanjakana
masi-mandidy, tokana, repoblikana ary tsy mampifangaro ny fitondram-panjakàna sy
ny finoana.\\

\noindent
« Repoblikan'i Madagasikara » no anarana entin'io Fanjakana io.\\

\noindent
Ny demokrasia sy ny feni-kevitra momba ny Fanjakana tan-dalàna no fototra
iorenan'ny Repoblika.\\

\noindent
Eo amin'ny faritry ny taniny no maha
masi-mandidy azy.\\

\noindent
Tsy misy mihitsy mahazo manohintohina ny maha iray tsy anombinana ny
tanin'ny Repoblika.\\

\noindent
Tsy azo amidy na atakalo ny tanim-pirenena.\\

\noindent
Ny fomba sy fepetra fivarotana sy fampanofana tany ho an'ny vahiny dia
voafaritry ny lalàna.

\andininy{}Ny Fanjakana dia manamafy ny tsy fombàny ny atsy sy ny aroa eo
anatrehan'ny finoana samy hafa.\\

\noindent
Ny tsy fampifangaroana ny fitondram-panjakana sy ny finoana eto amin'ny
Repoblika dia miankina amin'ny foto-kevitra mikasika ny fisarahan'ny
raharaham-panjakana sy ny fikambanam-pivavahana ary ny solontenany.\\

\noindent
Ny Fanjakana sy ny fikambanam-pivavahana dia tena tsy mahazo mifampitsabatsabaka
eo amin'ny sehatrasan'izy ireo tsirairay avy.\\

\noindent
Tsy misy mihitsy Lehiben'Andrim-panjakana na mambra ao amin'ny Governemanta
afaka miditra any amin'ny ambaratongam-pitarihana ny fikambanam-pivavahana iray;
raha tsy izany dia azon'ny Fitsarana Avo momba ny Lalàmpanorenana atao ny
manongana azy na heverina ho toy ny mametra-pialàna avy hatrany amin'ny asany na
ny raharahany izy.

\andininy{}Ny Repoblikan'i Madagasikara dia Firenena miankina amin'ny rafitry ny
Vondrombahoakam-paritra Itsinjaram-pahefana izay ahitana Kaominina, Faritra ary
Faritany ka ny fari-pahefana sy ny foto-kevitra momba ny fizakan-tena
ara-pitondran-draharaha sy ara-pitantanam-bola dia iantohan'ny Lalàmpanorenana
sy faritan'ny Lalàna.

\andininy{}« \emph{Fitiavana – Tanindrazana – Fandrosoana} » no filamatry ny
Repoblikan'i Madagasikara.\\

\noindent
Ny fanevany dia saina telo soratra, fotsy, mena, maitso, vita amin'ny tsivalana
telo mahitsizoro mitovy refy, ka ny voalohany fotsy ary mitsangana manaraka ny
tahon-tsaina, ny roa hafa mandry ka ny mena ambony ary ny maitso ambany.\\

\noindent
Ny teny malagasy no tenim-pirenena.\\

\noindent
« Ry Tanindrazanay malala ô » no hiram-pirenena.\\

\noindent
Antananarivo no Renivohitry ny Repoblikan'i Madagasikara.\\

\noindent
Ny lalàna no mametra ny fitombokasem-panjakana sy ny mari-piandrianan'ny
Firenena.\\

\noindent
Ny teny malagasy sy frantsay no teny ofisialy.

\andininy{}Ny vahoaka izay ipoiran'ny fahefana rehetra no masi-mandidy ka
mampiasa izany amin'ny alàlan'ny solontenany lany amin'ny alàlan'ny fifidianana
andraisan'ny rehetra anjara mivantana na tsy mivantana, na koa amin'ny alalan'ny
fitsapan-kevi-bahoaka. Tsy misy mihitsy ampahany amin'ny vahoaka, na olon-tokana
mahazo manendry tena hampiasa izany fahefana feno izany.\\

\noindent
Ny fandaminana sy ny fitantanana ny raharaha rehetra momba ny fifidianana dia
ankinina amin'ny rafitra mahaleotena iray eto amin'ny Firenena.\\

\noindent
Ny lalàna no mandamina ny fombafomba fampandehanana io rafitra io.\\

\noindent
Mpifidy avokoa, araka ny fepetra faritan'ny lalàna, ny olom-pirenena lahy sy
vavy rehetra mizaka ny zon'ny isam-batan'olona sy ny zo politika.
Didim-pitsarana raikitra no hany mahavery ny zo maha-mpifidy.

\andininy{}Ny lalàna dia maneho ny safidim-bahoaka. Mitovy ny olona rehetra eo
anatrehan'ny lalàna, na izy natao hiarovana, na izy natao handidy, na izy natao
hamaizana.\\

\noindent
Mitovy zo ny olona rehetra eo anatrehan'ny lalàna, samy manana fahalalahana
fototra arovan'ny lalàna ary tsy misy fanavakavahana na amin'ny maha-lahy na
amin'ny maha-vavy, na amin'ny fari-pahalalàna, na amin'ny fari-piainana, na
amin'ny fiaviana, na amin'ny finoana, na amin'ny tsy fitovian-kevitra.\\

\noindent
Ny lalàna no manome tombony ny fitoviana amin'ny fidirana sy ny
fandraisan'anjaran'ny vehivavy sy ny lehilahy amin'ny fisahanan'asam-panjakana
sy amin'ny asa eo amin'ny sehatry ny fiainana ara-politika, ara-toekarena sy
ara-tsosialy.

\section{Ny amin'ny fahalalahana sy ny amin'ny zo aman'andraikitry ny
  olom-pirenena}
\label{sec:ny-aminny-fahal}

\subsection{Voalohany ny amin'ny zo aman'andraikitry ny isam-batan'olona sy ny
  zo politika}
\label{sec:voalohany-ny-aminny}



\andininy{}Ny zon'ny isam-batan'olona sy ny fahalalahana fototra dia iantohan'ny
Lalàmpanorenana ary ny lalàna no mandamina ny fampiasana azy.


\andininy{}Arovan'ny Lalàna ny zon'ny olona rehetra hiaina. Tsy azo atao ny
manala ain'olona amin'ny fomba tsy rariny. Tsy azo heverina ho fandikàna ity
andininy ity ny famonoana raha toa ka tsy maintsy ilaina ny fampiasana hery mba
hiantohana ny fiarovana ny olona rehetra amin'ny herisetra tsy ara-dalàna.\\

\noindent
Tsy misy olona azo ampahoriana, saziana na ampijaliana amin'ny fomba tafahoatra
sy tsy mendrika manetry ny maha-olombelona.\\

\noindent
Voarara indrindra ny manery olona iray hanaovana andrana ara-pitsaboana na
ara-tsiansa raha tsy nahazoana ny fanekeny malalaka.

\andininy{}Ny olona rehetra dia manan-jo amin'ny fahalalahana ary tsy azo
samborina na tànana am-ponja tsy amin'antony.\\

\noindent
Tsy misy olona azo enjehina na samborina na tànana am-ponja raha tsy noho ny
anton-javatra voafaritry ny lalàna ary araka ny fomba voadidiny.\\

\noindent
Izay niharan'ny fisamborana na fitànana am-ponja tsy ara-dalàna dia manan-jo
hahazo onitra.

\andininy{}Ny fahalalahana maneho hevitra sy miteny, ny fahalalahana eo amin'ny
fifandraisana, ny fahalalahana manao gazety, ny fahalalahana hiditra aminà
fikambanana na hanorina fikambanana, ny fahalalahana hivory malalaka, ny
fahalalahana mivezivezy, ny fahalalahana amin'ny fieritreretana sy amin'ny
finoana dia samy iantohana ho an'ny rehetra ary tsy azo ferana raha tsy hoe ho
fanajana ny fahalalahana sy ny zon'ny hafa sy noho ny fahaterena hiaro ny
filaminam-bahoaka, ny fahamendrehan'ny firenena ary ny filaminan'ny Fanjakàna.

\andininy{}Manan-jo hahalala vaovao ny tsirairay.\\

\noindent
Tsy misy faneriterena mialoha azo ampiharina amin'ny fampahalalam-baovao amin'ny
endriny rehetra ankoatr'ireo izay mamoafady na mety hanohintohina ny
filaminam-bahoaka.\\

\noindent
Zo ny fahalalahana amin'ny fampahalalam-baovao, na toy inona toy inona no
hanohanana izany. Ny fampiasana io zo io dia adidy sy andraikitra saingy
voafehin'ny fombafomba, fepetra, na sazy voalazan'ny lalàna, izay fepetra ilaina
ao anatin'ny fiaraha-monina demokratika. Voarara amin'ny endriny rehetra ny
sivana.\\

\noindent
Ny fisahanana ny asa aman-draharahan'ny mpanao gazety dia voalamina araka ny
lalàna.

\andininy{}Ny olom-pirenena malagasy rehetra dia manana zo handao sy hiverina
eto amin'ny tanim-pirenena araka ny fepetra voatondron'ny lalàna.\\

\noindent
Zon'ny tsirairay ny mivezivezy sy manorim-ponenana an-kahalalahana manerana ny
tanin'ny Repoblika, rehefa voahaja ny zon'ny hafa sy ny fepetra faritan'ny
lalàna.

\andininy{}Ny olona tsirairay dia iantohana amin'ny tsy fahazoana manao
an-keriny na herisetra amin'ny tenany, amin'ny fonenany sy amin'ny
tsiambaratelon'ny fifandraisany.\\

\noindent
Tsy misy fisavan-trano na toerana azo atao raha tsy alàlana omen'ny lalàna ary
araka ny baiko an-tsoratra nomen'ny fahefana mpitsara afaka manao izany, afa-tsy
amin'ireo izay tratra ambodiomby.\\

\noindent
Tsy misy olona azo sazina raha tsy araka ny lalàna navoaka hanan-kery alohan'ny
nanaovana ilay fihetsika mahavoasazy.\\

\noindent
Tsy misy olona azo sazina indroa noho ny toe-javatra iray efa nanamelohana
azy. Ny lalàna dia miantoka ny zo hitady ny rariny sy ny hitsiny ho an'ny olona
rehetra ary tsy vato misakana izany velively ny tsy fahampian'ny fidiram-bola
aminy.Iantohan'ny Fanjakana ny zom-piarovan-tena feno sy tsy azo hozongozonina
eo anatrehan'ny antokom-pitsarana rehetra, amin'ny dingana rehetra eo amin'ny
fizotry ny ady, manomboka hatrany amin'ny famotorana mialoha eo amin'ny
ambaratongan'ny mpanao famotorana na ny fampanoavana.\\

\noindent
Voarara ny fanerena ara-tsaina rehetra sy/na fampijaliana ara-batana mba
hisamborana olona iray na hitànana azy am-ponja.\\

\noindent
Heverina ho tsy manan-tsiny ny voampanga sy voarohirohy rehetra mandra-pisian'ny
didim- pitsarana raikitra manameloka azy.\\

\noindent
Toe-javatra manokana ny fitànana am-ponja vonjimaika.

\andininy{}Zon'ny olona rehetra no mivondrona an-kahalalahana ao anaty
fikambanana, nefa tsy maintsy manaraka ny voalazan'ny lalàna.\\

\noindent
Toy izany koa ny momba ny zo hanangana antoko politika. Ny lalàna momba ny
antoko politika no mametra ny fomba entina manangana sy mamatsy vola azy ireo.\\

\noindent
Raràna ny fikambanana sy ny antoko politika izay manohintohina ny maha-iray tsy
mivaky ny Firenena sy ny foto-kevitra repoblikana, sy izay manindrahindra
fanjakazakana na fanavakavahana ara-pirazanana, ara-poko na ara-pinoana.\\

\noindent
Mandray anjara amin'ny fanehoana safidy ny antoko sy fikambanana politika.\\

\noindent
Ny Lalàmpanorenana no miantoka ny zo hijoro ho mpanohitra amin'ny fomba
demokratika.\\

\noindent
Aorian'ny fifidianana solombavambahoaka, ny vondrona politika mpanohitra dia
manendry ny lehiben'ny mpanohitra. Raha tsy misy ny fifanarahana dia ny
lehiben'ny vondrona politika mpanohitra izay nahazo vato manan-kery be indrindra
tamin'ny latsabato no heverina ho lehiben'ny mpanohitra.\\

\noindent
Ny sata mifehy ny mpanohitra sy ny antoko mpanohitra, eken'ity Lalàmpanorenana
ity sy manome azy indrindra ny sehatra ahafahana haneho hevitra, dia faritana
amin'ny alàlan'ny lalàna.

\andininy{}Manan-jò hilatsaka hofidina amin'ny fifidianana rehetra voalazan'ny
Lalàmpanorenana ny olom-pirenena tsirairay rehefa voahaja ny fepetra takian'ny
lalàna.

\andininy{}Adidin'ny tsirairay ny manaja ny Lalàmpanorenana sy ireo
Andrim-panjakana, ary ireo didy aman-dalàn'ny Repoblika eo am-pizakàna ny zo sy
ny fahalalahana eken'ity Lalampanorenana ity.

\subsection{Ny amin'ny zo sy adidy momba ny toekarena sy sosialy ary koltoraly}
\label{sec:ny-aminny-zo}

\andininy{}Arovana sy iantohan'ny Fanjakàna ny fampiasàna ny zo izay miantoka
ny maha-izy azy ny tsirairay, ny fahamendrehany, ny fivelarany feno ara-batana
sy ara-tsaina ary ara-pitondrantena.

\andininy{}Voninahitra sy adidy ny fanompoam-pirenena voadidin'ny lalàna.\\

\noindent
Ny fanatontosana izany dia tsy manembantsembana ny toerana misy ny olom-pirenena
eo amin'ny asany na ny fampiasana ny zo politika ananany.

\andininy{}Mankatò sy mandamina ny zon'ny tsirairay amin'ny fiarovana ny
fahasalamana dieny hatrany am-bohoka ny Fanjakana amin'ny alàlan'ny fandaminana
ny fitsaboana ara-panjakana maimaimpoana, ka ny maha maimaimpoana izany dia avy
amin'ny fahafaha-manao ny firaisankinam-pirenena.

\andininy{}Arovan'ny Fanjakàna ny ankohonana izay singa voajanahary sy
andrin'ny fiaraha-monina. Manana zo hanorina fianakaviana ary hamela ireo
fananany manokana holovain'ny taranany ny olona tsirairay avy.

\andininy{}Ny Fanjakana no miantoka amin'ny alàlan'ny famoahan-dalàna sy ny
fananganana rafitra mpiahy mifanentana amin'ny fiarovana ny fianakaviana mba
hahazoany mivoatra tsara, ka ao anatin'izany indrindra ny fiahiana ny reny sy ny
zaza.

\andininy{}Raisin'ny Fanjakana ho adidy ny fandraisana ny fepetra ilaina
amin'ny fampandrosoana ara-tsaina ny isam-batan'olona ka tsy misy fameperana
afa-tsy ny fahaizan'ny tsirairay.

\andininy{}Manan-jo hianatra sy handray fanabeazana ny ankizy tsirairay avy ka
tompon'andraikitra amin'izany ny ray aman-dreny ka hajaina ny fahalalahana
hisafidy ananany. Ny Fanjakana dia mandray ho adidy ny fampivoarana ny
fanofanana arak'asa.

\andininy{}Mandamina fanabeazam-bahoaka maimaimpoana sy takatry ny rehetra
hidirana ny Fanjakana. Ny fampianarana ambaratonga voalohany dia tsy maintsy
ataon'ny rehetra.

\andininy{}Eken'ny Fanjakana ny fisian'ny fampianarana tsy miankina aminy ary
iantohany ny fahalalahana hanome izany fampianarana izany raha toa ka mifandanja
amin'ny fepetra mikasika ny fampianarana momba ny fitandremam-pahasalamana sy ny
fitondran-tena ary ny lentan'ny fianarana faritan'ny lalàna.\\

\noindent
Ireny toeram-pampianarana tsy miankina ireny dia fehezin'ny lalàna momba ny
hetra araka ny fepetra faritan'ny lalàna.

\andininy{}Manan-jo handray anjara amin'ny fiainana ara-kolontsain'ny
fiaraha-monina, amin'ny fandrosoana ara-tsiansa ary hioty ireo vokatsoa
aterak'izany ny isam-batan'olona.\\

\noindent
Ny Fanjakana, miaraka amin'ny Vondrombahoakam-paritra itsinjaram-pahefana, no
miantoka ny fampivoarana sy ny fiarovana ny vakoka ara-kolontsaim-pirenena sy ny
famokarana ara-tsiansa, ara-haisoratra ary ara-javakanto.\\

\noindent
Ny Fanjakana, miaraka amin'ny Vondrombahoakam-paritra itsinjaram-pahefana no
miantoka ny zom-pitompoana ara-tsaina.

\andininy{}Zo sy adidy ho an'ny olom-pirenena tsirairay ny asa sy ny fiofanana
arak'asa.\\

\noindent
Ny fidirana amin'ny asam-panjakana dia misokatra ho an'ny olom-pirenena rehetra
tsy misy fepetra ankoatry ny fahafaha-manao sy ny fahaizana.\\

\noindent
Na izany aza anefa, azo atao ny mametra isaky ny fari-piadidiana, ny isan'ny
olona raisina ho ao amin'ny asam-panjakana ka ho faritan'ny lalàna ny fe-potoana
faharetan'izany sy ny fomba fampiharana azy.

\andininy{}Tsy misy natao ho matiantoka amin'ny asa aman-draharahany noho ny
maha-lahy na maha-vavy azy, ny taona, ny finoana, ny firehan-kevitra, ny
fiaviana, noho ny maha-mpikambana ao anaty rafitra sendikaly na koa ny
faharesen-dahatra ara-politika.

\andininy{}Ny olom-pirenena tsirairay dia manan-jo handray valin-kasasarana
mifandanja amin'ny asa ataony izay ahafahany miantoka azy sy ny ankohonany,
fiainana araka ny fahamendrehan'ny maha-olona.

\andininy{}Miezaka ny Fanjakana hiahy ny filàn'ny olom-pirenena rehetra izay
mety ho tsy afaka miasa noho ny taonany, na noho ny kilemany ara-batana na
ara-tsaina, ka izany dia amin'ny alalàn'ny firotsahan'ny andrim-panjakana na
antokon-draharaha miendrika sosialy an-tsehatra.

\andininy{}Eken'ny Fanjakana ny zon'ny mpiasa tsirairay hiaro ny tombontsoany
amin'ny alalàn'ny hetsika sendikaly ary indrindra amin'ny alalàn'ny fahalalahana
hanangana sendikà. Malalaka ny fidirana aminà sendika.

\andininy{}Zon'ny mpiasa tsirairay, indrindra amin'ny alàlan'ny solontenany, ny
mandray anjara amin'ny famaritana ireo fitsipika sy fepetra momba ny asa.

\andininy{}Ekena ny zo hanao fitokonana nefa izany dia tsy tokony hanohintohina
ny fampandehanana ny asam-bahoaka na ny tombontsoa fototry ny Firenena. Ny
fepetra hafa fampiasana io zo io dia feran'ny lalàna.

\andininy{}Iantohan'ny Fanjakana ny fananan'ny tsirairay zo hanana fananana
manokana. Tsy misy fananan'olona azo ongotana aminy raha tsy noho ny
tombontsoam-bahoaka kanefa izany dia tsy maintsy handoavana onitra ara-drariny
mialoha.

Ny Fanjakana no miantoka ny fanamorana amin'ny fahazoana ny fizakan-tany amin'ny
alàlan'ny famoronana lalàna sy rafitra hisian'ny mangarahara amin'ny
fahazoam-baovao momba ny tany.

\andininy{}Amorain'ny Fanjakàna ny fahazoan'ny olom-pirenena trano fonenana
amin'ny alàlan'ny fomba famatsiam-bola sahaza.

\andininy{}Tsy maintsy mandroso miandalàna sy tombanana arakaraky ny fahafahany
mandoa hetra ny fandraisan'ny olom-pirenena tsirairay anjara amin'ny
fandaniam-bolam-panjakana.

\andininy{}Ny Fanjakana no miantoka ny fahalalahan'ny fandraharahana ao
anatin'ny faritry ny fanajana ny tombontsoa iombonana, ny filaminam-bahoaka, ny
fitondran-tena mendrika ary ny tontolo iainana.

\andininy{}Ny Fanjakana no miantoka ny fiarovana ny renivola sy ny
fampiasam-bola hamokarana.

\andininy{}Ny Fanjakana no miantoka ny tsy fiandaniana ara-politika amin'ny
Fitondran-draharaham-panjakana, ny Foloalindahy, ny Fitsarana, ny Mpitandro ny
filaminana, ny Fampianarana ary ny Fanabeazana.\\

\noindent
Izy no mandamina ny Fitondran-draharaham-panjakana mba hisorohana amin'ireo asa
rehetra fandanindaniam-poana sy ny fanodinkodinana ny volam-panjakana ho an'ny
tombontsoa manokana na politika.

\section{Ny amin'ny fandaminana ny fanjakana}
\label{sec:ny-aminny-fand}

\andininy{}Ireto avy ny Andrim-panjakana~:
\begin{itemize}
\item ny Filohan'ny Repoblika sy ny Governemanta~;
\item ny Antenimierampirenena sy ny Antenimierandoholona~;
\item ny Fitsarana Avo momba ny Lalàmpanorenana.
\end{itemize}

\noindent
Ny Fitsarana Tampony, ny Fitsarana Ambony sy ireo fitsarana miankina aminy ary
koa ny Fitsarana Avo no manatanteraka ny asam-pitsarana.


\andininy{}Ny lalàna no mamaritra ny habetsahana, ny fepetra ary ny fombafomba
fanomezana ny tambin-karama omena ireo olo-manan-kaja nantsoina hisahana
andraikitra maha olom-boafidy, na hamita asa, na hanatontosa iraka eo
anivon'ireo Andrim-panjakana voalazan'ity Lalàmpanorenana ity.\\

\noindent
Alohan'ny hanatontosana ny asa na iraka sy hisahanana ny andraikiny dia
mametraka fanambaram-pananana manoloana ny Fitsaràna Avo Momba ny
Lalàmpanorenana ireo sokajin'olona voalazan'ny andàlana etsy ambony.\\

\noindent
Tsy misy olona amin'ireo voalazan'ny andininy faha-40 mahazo mandray ivelan'ny
zony tamby na karama avy amin'olona na rafitra mizaka zo aman'andraikitry ny
isam-batan'olona, vahiny na teratany, raha toa izany manakantsakana ny
fanatontosany ara-dalàna ny andraikitra nomena azy fa raha tsy izany dia
aongana.\\

\noindent
Ny lalàna no mamaritra ny fomba fampiharana ireo fepetra ireo indrindra ny
amin'ny famerana ny zo, ny tamby sy ny karama ary ny paika arahina amin'ny
fanonganana.

\andininy{}Ny asa sahanina eny anivon'ny Andrim-panjakana dia tsy tokony ho zary
loharanon-karena tsy ara-drariny na fitaovana ampiasaina ho amin'ny
tombontsoan'ny tena manokana.

\andininy{}Ny Filankevitra Ambony ho amin'ny Fiarovana ny Demokrasia sy ny
Fanjakana tan-dalàna no miandraikitra ny fanaraha-maso ny fanajana ny toetra
fototry ny fahefana, ny demokrasia sy ny fanajàna ny Fanjakana tan-dalàna, ny
fanaraha-maso ny fampiroboroboana sy ny fiarovana ny zon'olombelona.\\

\noindent
Ny lalàna no mamaritra ny fombafomba mikasika ny ho anisany sy ny fandaminana
ary ny fampandehanana ny Filankevitra Ambony.

\subsection{Ny amin'ny mpanatanteraka}
\label{sec:mpanantanteraka}

\andininy{}Ny asan'ny mpanatanteraka dia sahanin'ny Filohan'ny Repoblika sy ny
Governemanta.

\subsubsection{Ny amin'ny Filohan'ny Repoblika}
\label{sec:ny-aminny-filohanny}

\andininy{}Ny Filohan'ny Repoblika no Filoham-panjakana.\\

\noindent
Fidina hiasa mandritra ny dimy taona azo havaozina indray mandeha monja, amin'ny
alàlan'ny fifidianana andraisan'ny rehetra anjara mivantana izy.\\

\noindent
Izy no miantoka, amin'ny fanelanelanany, ny fampandehanana ara-dalàna sy mitohy
ny fahefam-panjakana, ny fahaleovantenam-pirenena sy ny maha iray tsy anombinana
ny tanindrazana.  Izy no mitandro ny fiarovana sy ny fanajana ny
fiandrianam-pirenena na eto an-toerana na any ivelany. Izy no miantoka ny
Firaisam-pirenena.\\

\noindent
Ny Filohan'ny Repoblika no miantoka ireo asa ireo ao anatin'ny fahefana izay
napetrak'ity Lalàmpanorenana ity aminy.

\andininy{}Izay olona rehetra milatsaka hofidina ho Filohan'ny Repoblika dia tsy
maintsy mizaka ny zom-pirenena Malagasy, misitraka ny zon'ny isam-batan'olona sy
ara-politika, feno dimy amby telopolo taona farafahakeliny amin'ny vaninandro
hifaranan'ny fametrahana ny filatsahana hofidina, monina eto amin'ny tanin'ny
Repoblikan'i Madagasikara enim-bolana farafahakeliny mialohan'ny vaninandro
farany ferana amin'ny fametrahana ny filatsahan-kofidina.\\

\noindent
Ny Filohan'ny Repoblika amperin'asa izay milatsaka hofidina amin'ny fifidianana
ho Filohan'ny Repoblika dia mametra-pialàna amin'ny toerany enimpolo andro
alohan'ny hanatanterahana ny fifidianana Filohan'ny Repoblika. Amin'io
anton-javatra io, ny Filohan'ny Antenimierandoholona no misahana ny anjara
raharahan'ny Filohan'ny Repoblika andavanandro mandra-panolorana ny fahefana
amin'ny Filoha vaovao.\\

\noindent
Raha toa ny Filohan'ny Antenimierandoholona ka milatsaka hofidina ihany koa, dia
iarahan'ny Governemanta misahana ny asan'ny Filoham-panjakana.\\

\noindent
Voarara ho an'ireo olo-manan-kaja rehetra misahana andraikitra maha-olom-boafidy
na manatanteraka asa eo anivon'ny Andrim-panjakana ary milatsaka hofidina
amin'ny fifidianana ho Filohan'ny Repoblika, ny mampiasa mba hanaovana
fampielezan-kevitra ny fitaovana na tombontsoa izay ananany noho ny asa
ataony. Ny fandikana izany izay voazahan'ny Fitsarana Avo momba ny
Lalàmpanorenana fototra dia zary antony tsy hampanan-kery ny filatsahana
hofidina.

\andininy{}Ny fifidianana ny Filohan'ny Repoblika dia tanterahina telopolo
andro farafahakeliny ary enimpolo andro farafahabetsany alohan'ny fiafaran'ny
fe-potoana fiasan'ny Filoha am-perinasa.\\

\noindent
Ireo fe-potoana ireo dia mihatra aorian'ny fahitana fototra ny
fahabangan-toerana ambaran'ny Fitsarana Avo momba ny Lalàmpanorenana amin'ireo
toe-javatra voalazan'ny andininy faha-52 sy faha-132 amin'ity Lalàmpanorenana
ity.\\

\noindent
Voafidy ho Filohan'ny Repoblika avy hatrany izay mahazo ny antsamanilan'ny vato
manan-kery amin'ny fihodinana voalohany. Raha tsy tratra izany, dia asiana
fihodinana faharoa ka izay mahazo ny ampahany be indrindra amin'ny vato
manan-kery no lany ho Filohan'ny Repoblika amin'ireo mpilatsa-kofidina roa
nahazo vato be indrindra tamin'ny fihodinana voalohany. Ny fihodinana faharoa
dia atao telopolo andro raha ela indrindra aorian'ny fanambarana ofisialy ny
vokatry ny fihodinana voalohany.\\

\noindent
Raha misy ny fahafatesan'ny mpilatsaka hofidina alohan'ny fihodinana iray
amin'ny fandatsaham-bato, na koa misy toe-javatra hafa tsy azo anoharana
voamarin'ny Fitsarana Avo momba ny Lalàmpanorenana ara-dalàna, dia ahemotra
amin'ny vaninandro vaovao ny fifidianana araka ny fepetra sy araky ny fombafomba
izay ho faritana amin'ny alàlan'ny lalàna fehizoro.\\

\noindent
Ny Filoha am-perinasa tsy nilatsaka hofidina amin'ny fifidianana dia mbola
manohy ny asany mandra-pahatongan'ny fotoana hanolorana fahefana ny mpandimby
azy araka ny fepetra voalazan'ny andininy faha-48.

\andininy{}Ny fanoloram-pahefana ofisialy dia atao eo amin'ny Filoha teo aloha
sy ny Filoha voafidy vaovao.\\

\noindent
Alohan'ny handraisany ny asany, ny Filohan'ny Repoblika, amin'ny
fotoam-pitsarana manetriketrika ataon'ny Fitsarana Avo momba ny Lalàmpanorenana,
eo anoloan'ny Firenena sy eo anatrehan'ny Governemanta, ny Antenimierampirenena,
ny Antenimierandoholona ary ny Fitsarana Tampony, dia manao izao fianianana
izao~:
\begin{quoting}[begintext=«~, font=itshape, endtext=~»]
  Eto anatrehan'Andriamanitra Andriananahary sy ny Firenena ary ny Vahoaka,
  mianiana aho fa hanatanteraka antsakany sy andavany ary amim-pahamarinana ny
  andraikitra lehibe maha-Filohan'ny Firenena Malagasy ahy.

  Mianiana aho fa hampiasa ny fahefana natolotra ahy ary hanokana ny heriko
  rehetra hiarovana sy hanamafisana ny firaisam-pirenena sy ny zon'olombelona.

  Mianiana aho fa hanaja sy hitandrina toy ny anakandriamaso ny Lalàmpanorenana
  sy ny lalàm-panjakana, hikatsaka hatrany ny soa ho an'ny Vahoaka Malagasy tsy
  ankanavaka.
\end{quoting}
\noindent
Manomboka ny andro nanaovana ny fianianana ny fe-potoana iasan'ny Filohan'ny
Repoblika.

\andininy{}Ny asan'ny Filohan'ny Repoblika dia tsy azo ampirafesina aminà
asam-panjakana nahavoafidim-bahoaka, asa aman-draharaha hafa, asa ao anatin'ny
antoko politika, vondrona politika, na fikambanana, ary fisahanana andraikitra
eo anivon'ny fikambanam-pivavahana.\\

\noindent
Izay rehetra fandikana ireo fepetra voalazan'ity andininy ity, voamarin'ny
Fitsarana Avo momba ny Lalàmpanorenana, dia zary ho antony amin'ny tsy
fahafahana miasa tanteraka ho an'ny Filohan'ny Repoblika.

\andininy{}Ny tsy fahafahana miasa miserana manjo ny Filohan'ny Repoblika noho
ny tsy fahafahana ara-batana na ara-tsaina hisahana ny asany izay voamarina
ara-dalàna dia ambaran'ny Fitsarana Avo momba ny Lalàmpanorenana, rehefa
nampakaran'ny Antenimierampirenena ny raharaha, araka ny fanapahan-kevitra
nolanian'ny roa ampahatelon'ny mpikambana ao aminy.\\

\noindent
Raha sendra misy tsy fahafahana miasa miserana dia ny Filohan'ny
Antenimierandoholona no misahana vonjimaika ny asan'ny Filoham-panjakana.

\andininy{}Ny fanalàna ny tsy fahafahana miasa miserana dia tapahan'ny
Fitsarana Avo momba ny Lalàmpanorenana, rehefa nanao fampakaran-draharaha ny
Antenimiera.\\

\noindent
Ny tsy fahafahana miasa miserana dia tsy tokony hihoatry ny telo volana, ka
aorian'izay dia azon'ny Fitsarana Avo momba ny Lalàmpanorenana atao ny manambara
ny amin'ny fanovàna ny tsy fahafahana miasa miserana ho tsy fahafahana miasa
tanteraka aorian'ny fampakaran-draharaha momba izany ataon'ny Antenimiera roa
tonta izay manapaka amin'ny latsabato misaraka ary nolanian'ny roa
ampahatelon'ny mpikambana ao aminy avy.

\andininy{}Aorian'ny fametraham-pialàna, fandaozana ny fahefana na amin'ny
endriny inona na amin'ny endriny inona, fahafatesana, ny tsy fahafahana miasa
tanteraka na ny fanonganana nambara, ny fahabangan'ny toeran'ny Filohan'ny
Repoblika dia hozahan'ny Fitsarana Avo momba ny Lalàmpanorenana fototra.\\

\noindent
Raha vao hita fototra ny fahabangan'ny toeran'ny Filohan'ny Repoblika, dia ny
Filohan'ny Antenimierandoholona no misahana ny asan'ny Filoham-panjakana.\\

\noindent
Raha misy tsy fahafahana miasa manjo ny Filohan'ny Antenimierandoholona hita
fototry ny Fitsarana Avo momba ny Lalàmpanorenana, dia iarahan'ny Governemanta
manontolo misahana ny asan'ny Filoham-panjakana.

\andininy{}Aorian'ny fizahana fototra ataon'ny Fitsarana Avo momba ny
Lalàmpanorenana ny fahabangan'ny toeran'ny Filohan'ny Repoblika, dia hisy ny
fifidianana Filohan'ny Repoblika vaovao atao ao anatin'ny 30 andro
farafahakeliny ary 60 andro farafahabetsany, araka ny fepetra voalazan'ny
andininy faha-46 sy faha-47 amin'ny Lalàmpanorenana.\\

\noindent
Mandritry ny fe-potoana izay miantomboka amin'ny fahitana fototra ny
fahabangan-toerana ka hatramin'ny fotoana hanolorana ny fahefana amin'ny
Filohan'ny Repoblika vaovao na amin'ny fanesorana ny tsy fahafahana miasa
miserana dia tsy azo ampiharina ny andininy faha-60, faha-100, faha-103,
faha-162 ary faha-163 amin'ny Lalàmpanorenana.

\andininy{}Ny Filohan'ny Repoblika no manendry ny Praiminisitra araka ny
tolotry ny antoko na vondrona antoko maro an'isa ao amin'ny
Antenimierampirenena.\\

\noindent
Ny Filohan'ny Repoblika no mampitsahatra ny Praiminisitra amin'ny asany,
aorian'ny fanolorany ny fametraham-pialàn'ny Governemanta, na vokatry ny
fahadisoana goavana, na noho ny tsy fahombiazana azo tsapain-tànana.\\

\noindent
Ny Filohan'ny Repoblika no manendry ny mambra ao amin'ny Governemanta sy
mampitsahatra azy ireo amin'ny asany, araka ny tolokevitra avy amin'ny
Praiminisitra.

\andininy{}Ny Filohan'ny Repoblika no~:

\begin{enumerate}
\item mitarika ny Filankevitry ny Minisitra~;

\item manao sonia ny hitsivolana noraisina teo amin'ny Filankevitry ny Minisitra
  araka ireo toe-javatra sy fepetra voalazan'ity Lalàmpanorenana ity~;

\item manao sonia ny didim-panjakana notapahina teo anivon'ny Filankevitry ny
  Minisitra~;

\item manendry, eo anivon'ny Filankevitry ny Minisitra, amin'ny asam-panjakàna
  ambony izay ferana amin'ny alàlan'ny didim-panjakana raisina eo amin'ny
  Filankevitry ny Minisitra ny lisitra~;

\item afaka manapaka, eo amin'ny Filankevitry ny Minisitra, momba ny raharaha
  rehetra mavesan-danja mikasika ny Firenena, ny amin'ny hanontaniana mivantana
  ny hevitry ny vahoaka amin'ny alàlan'ny fitsapan-kevi-bahoaka~;

\item manoritra sy manapaka, eo amin'ny Filankevitry ny Minisitra, ny politika
  ankapobe arahin'ny Fanjakana.

\item manara-maso ny fampiharana ny politika ankapoben'ny Fanjakàna voafaritra
  araka izany sy ny asan'ny Governemanta~;

\item mampiasa ireo ratsa-mangaika mpanara-maso ny
  Fitondran-draharaham-panjakana.
\end{enumerate}

\noindent
Afaka mamindra ny sasantsasany amin'ny fahefany amin'ny Praiminisitra ny
Filohan'ny Repoblika.

\andininy{}Ny Filohan'ny Repoblika no Filoha Faratampon'ny Foloalindahy ka izy
no miantoka ny maha-iray tsy mivaky azy. Amin'izany dia ampian'ny Filankevitra
Ambony momba ny Fiarovam-pirenena izy.\\

\noindent
Ny Filankevitra Ambony momba Fiarovam-pirenena, eo ambany fahefan'ny Filohan'ny
Repoblika, dia miandraikitra indrindra ny fitandroana ny fandriampahalemana sy
ny fandrindrana ny asa nankinin'ny Foloalindahy aminy mba hitandroana ny
fandriampahalemana eo amin'ny fiaraha-monina. Ny fandaminana izany sy ny anjara
raharahany dia faritana amin'ny alàlan'ny lalàna.\\

\noindent
Ny Filohan'ny Repoblika no manapaka mandritra ny Filankevitry ny Minisitra ny
handefasana miaramila sy fitaovana amin'ny firotsahana any ivelany rehefa avy
naka ny hevitry ny Filankevitra Ambony momba ny Fiarovam-pirenena sy ny
Antenimiera.\\

\noindent
Izy no manapaka ao amin'ny Filankevitry ny Minisitra ny tetika momba ny
fiarovam-pirenena eo amin'ny lafiny miaramila, toekarena, sosialy, kolontsaina,
lafin-tany ary tontolo iainana.\\

\noindent
Ny Filohan'ny Repoblika no manendry ny miaramila nantsoina hisolotena ny
Fanjakana any amin'ireo antokon-draharaha iraisam-pirenena.

\andininy{}Ny Filohan'ny Repoblika no manendry sy mampody ireo Masoivoho sy
iraka manokan'ny Repoblika any amin'ireo Firenena sy Fikambanana
Iraisam-pirenena hafa.\\

\noindent
Izy no mandray ny taratasy fanendrena sy fampodiana ny solontenan'ny Fanjakana
sy ny Fikambanana Iraisam-pirenena ankatoavin'ny Repoblikan'i Madagasikara.

\andininy{}Ny Filohan'ny Repoblika no manana fahefana hamindra fo amin'ireo
voasazy.\\

\noindent
Izy no manolotra ny mari-boninahitra sy ny fampisaloram-boninahitry ny
Repoblika.

\andininy{}Ny Filohan'ny Repoblika no mamoaka hampanan-kery ny lalàna ao
anatin'ny telo herinandro manaraka ny fampitàn'ny Antenimierampirenena ny lalàna
nolaniana tanteraka.\\

\noindent
Alohan'ny fahataperan'io fe-potoana io, ny Filohan'ny Repoblika dia afaka
mangataka amin‘ny Antenimiera ny fandinihana indray ilay lalàna na ny andininy
sasantsasany ao aminy. Tsy azo làvina io fandinihina vaovao io.

\andininy{}Azon'ny Filohan'ny Repoblika atao ny mandrava ny
Antenimierampirenena rehefa avy nampahafantatra ny Praiminisitra, sy naka ny
hevitr'ireo Filohan'ny Antenimiera.\\

\noindent
Ny fifidianana ankapobe dia tanterahina ao anatin'ny enimpolo andro
farafahakeliny ary sivifolo andro farafahabetsany aorian'ny nanambarana ny
fandravàna.\\

\noindent
Ny Antenimierampirenena dia mivory avy hatrany ny alakamisy faharoa manaraka ny
fifidianana azy. Raha toa io fivoriana io ka atao ivelan'ny fe-potoana tokony
hanatanterahana fivoriana ara-potoana, dia hisokatra avy hatrany ary tanterahina
mandritra ny dimy ambinifolo andro ilay fotoam-pivoriana.\\

\noindent
Tsy misy fandravàna vaovao azo atao ao anatin'ny roa taona manaraka ireo
fifidianana ireo.

\andininy{}Raha tandindomin-doza ny Andrim-panjakan'ny Repoblika, ny
fahaleovantenam-pirenena, ny firaisam-pirenena na ny maha-iray tsy anombinana ny
Tanindrazana ka voatohintohina ny fizotra ara-dalànan'ny asan'ireo
fahefam-panjakana dia azon'ny Filohan'ny Repoblika atao ny manambara
ampahibemaso ny fisian'ny toe-javatra mampihotakotaka amin'ny faritra sasany na
eo amin'ny Firenena manontolo, izany hoe ny fotoam-pahamaizana, ny
fahalatsahan'ny Firenena an-katerena na ny fampiharana lalàna miaramila. Eo
amin'ny Filankevitry ny Minisitra no andraisan'ny Filohan'ny Repoblika ny
fanapahana amin'izany rehefa naka ny hevitry ny Filohan'ny Antenimierampirenena,
ny Filohan'ny Antenimierandoholona, ny Filohan'ny Fitsarana Avo momba ny
Lalàmpanorenana izy.\\

\noindent
Ny fanambarana ampahibemaso ny fisian'ny toe-javatra mampihotakotaka dia manome
fahefana manokana ny Filohan'ny Repoblika ka lalàna fehizoro no mamaritra ny
farafetrany sy ny fahelany.\\

\noindent
Raha vao nambara ampahibemaso fa misy ny toe-javatra mampihotakotaka dia azon'ny
Filohan'ny Repoblika atao ny manao lalàna amin'ny alàlan'ny hitsivolana.

\andininy{}Ny didy amam-pitsipika noraisin'ny Filohan'ny Repoblika, ankoatr'ireo
toe-javatra voalazan'ny andininy faha-54 andàlana voalohany sy faharoa, faha-58
andàlana voalohany sy faharoa, faha-59, faha-81, faha-60, faha-94, faha-100,
faha-114, faha-117 ary faha-119 dia iarahany manao sonia amin'ny Praiminisitra
ary, raha ilaina, iarahana amin'ny Minisitra voakasik'izany.

\subsubsection{Ny amin'ny Governemanta}
\label{sec:ny-aminny-govern}

\andininy{}Ny Governemanta dia ahitana ny Praiminisitra sy ny Minisitra.\\

\noindent
Izy no manatanteraka ny politika ankapoben'ny Fanjakana.\\

\noindent
Izy no tompon'andraikitra eo anoloan'ny Antenimierampirenena araka ny fepetra
voalazan'ny andininy faha- 100 sy faha-103 etsy ambany.\\

\noindent
Ny Governemanta no mampiasa ny Fitondran-draharaham-panjakana.

\andininy{}Ny asan'ny mambra ao amin'ny Governemanta dia tsy azo ampirafesina
amin'ny fisahanana ny asa maha-voafidim-bahoaka, ny andraikitra fisoloan-tena
amin'ny asa, ny fisahanana asa eo anivon'ny fikambanam-pivavahana, ny
fisahanan'asam-panjakana rehetra na ny asa aman-draharaha rehetra
andraisan-karama.\\

\noindent
Ny mambra rehetra ao amin'ny Governemanta, milatsaka hofidina amin'ny asa
maha-voafidim-bahoaka, dia tsy maintsy mametra-pialàna amin'ny asany raha vao
nambara fa azo raisina ny filatsahany hofidina.

\andininy{}Ny Praiminisitra, Lehiben'ny Governemanta no~:

\begin{enumerate}
\item mitarika ny politika ankapoben'ny Fanjakana~;

\item manana fahefana eo amin'ireo mambra ao amin'ny Governemanta izay izy no
  mitarika ny asa, sy tompon'andraikitra amin'ny fandrindrana ny asan'ny
  minisitera ary koa ny fanatanterahana ny fandaharanasam-pirenena rehetra momba
  ny fampandrosoana~;

\item manolotra volavolan-dalàna~;

\item manapaka ny volavolan-dàlana aroso hodinihin'ny Filankevitry ny Minisitra
  ary apetraka eo amin'ny biraon'ny iray amin'ireo Antenimiera roa tonta~;

\item miantoka ny fanatanterahana ny lalàna~;

\item manana ny fahefana hanao didy amam-pitsipika na dia eo aza ny fepetra
  voalazan'ny andininy faha-55 andàlana faha-3;

\item manara-maso ny fampiharana ireo didim-pitsarana~;

\item mampakatra raharaha, raha ilaina, any amin'ny Fisafoan-draharaha
  Ankapoben'ny Fanjakana sy ireo rantsa-mangaika hafa mpanara-maso ny
  Fitondran-draharaham-panjakana sy manao izay hampandeha tsara ny
  raharaham-panjakana, ny fitantanana arak'izay tokony ho izy ny volan'ny
  vondrombahoakam-panjakana sy ny an'ireo antokon-draharaham-panjakana~;

\item miantoka ny filaminana, ny fandriampahalemana ary ny maha-marin-toerana
  manerana ny tanim-pirenena anatin'ny fanajana ny firaisam-pirenena~;
  amin'izany, izy no mampiasa ireo hery rehetra miandraikitra ny filaminana, ny
  fitandroana ny fandriampahalemana, ny filaminana anatiny ary ny fiarovana~;

\item raha misy korontana politika goavana ary alohan'ny hanambaràna fisian'ny
  toe-javatra mampihotakotaka, dia afaka mampiasa ny mpitandro filaminana mba
  hamerina amin'ny laoniny ny fandriampahalemana rehefa avy naka ny hevitry ny
  manam-pahefana ambony ao amin'ny polisy, ny zandarimaria ary ny tafika, ny
  Filan-kevitra Ambony momba ny Fiarovam-pirenena sy ny Filohan'ny Fitsarana Avo
  momba ny Lalampanorenana,

\item Lehiben'ny fitondran-draharaham-panjakana~;

\item manendry amin'ny fisahanan'asa sivily sy miaramila ary koa amin'izay
  antokon-draharaha miankina amin'ny Fanjakana, na dia eo aza ny fepetra
  voalazan'ny andininy faha-55 andàlana faha-4.
\end{enumerate}
\noindent
Azony atao mamindra amin'ireo mpikambana ao amin'ny Governemanta ny ampahany
amin‘ny fahefany.\\

\noindent
Izy no misahana ny fampandrosoana mifandanja sy mirindra eo amin'ny
Vondrom-bahoaka Itsinjaram-pahefana rehetra.\\

\noindent
Tsy tohinina ny fepetra voalazan'ny andininy faha-55, nefa raha misy antony
manokana, dia azony atao, ny mitarika ny Filankevitry ny Minisitra araka ny
fanomezam-pahefana mazava tsara nomen‘ny Filohan'ny Repoblika sy araka ny
lahadinika voafaritra.

\andininy{}Ny Praiminisitra no mitarika ny Filankevitry ny Governemanta. Eo
amin'ny Filankevitry ny Governemanta izy no~:
\begin{enumerate}
\item mamaritra ny fampiharana ny politika ankapoben'ny Fanjakana sy manapaka ny
  fepetra ho raisina hiantohana ny fanatanterahana izany~;

\item misahana ny anjara raharaha hafa izay tsy maintsy hakàna ny hevitry ny
  Governemanta araka ny voalazan‘ity Lalàmpanorenana ity sy ireo lalàna
  manokana~;

\item manapaka ireo fepetra fampiharana ny fandaharanasam-pirenena momba ny
  fampandrosoana ara-toekarena sy ara-tsosialy ary koa ny amin'ny fanajariana ny
  tany, izay novolavolaina miaraka amin'ny manampahefana any amin'ireo
  Vondrom-bahoaka Itsinjaram-pahefana.
\end{enumerate}

\andininy{}Ny didy amam-pitsipika raisin'ny Praiminisitra dia iarahany manao
sonia amin'ny Minisitra miandraikitra ny fanatanterahana izany, raha ilaina.

\subsection{Ny amin'ny mpanao lalana}
\label{sec:ny-aminny-mpanao}


\andininy{}Ny Antenimierampirenena sy ny Antenimierandoholona no mandrafitra
ny Antenimiera. Izy no mandany ny lalàna sy manara-maso ny asan'ny Governemanta
ary ny fanaovana tombana ny politikam-panjakana.

\subsubsection{Ny amin'ny Antenimierampirenena}
\label{sec:ny-aminny-anten}

\andininy{}Fidin'ny daholobe hiasa mandritry ny dimy taona amin'ny alàlan'ny
latsabato mivantana ny mpikambana ao amin'ny Antenimierampirenena ka izay mahazo
vato be indrindra no lany.\\

\noindent
Ny fomba fanatanterahana ny latsabato dia ferana amin'ny alàlan'ny lalàna
fehizoro.\\

\noindent
Ny mpikambana ao amin'ny Antenimierampirenena dia mitondra ny anarana hoe~: «
Solombavambahoakan'i Madagasikara ».

\andininy{}Didim-panjakana raisina eo amin'ny Filankevitry ny Minisitra no
mametra ny isan'ny mpikambana ao amin'ny Antenimierampirenena, ny fitsinjarana
toerana manerana ny tanim-pirenena sy ny fizarazarana ny fari-pifidianana.

\andininy{}Tsy azo ampirafesina amin'ny andraikitra hafa
maha-olom-boafidim-bahoaka sy izay mety ho fisahanana asam-panjakana hafa
ankoatra ny asa fampianarana ny andraikitry ny solombavambahoaka.\\

\noindent
Miato avy hatrany amin'ny maha olom-boafidy azy ny solombavambahoaka voatendry
ho mpikambana ao amin'ny Governemanta. Ny mpisolo toerana no misolo azy.\\

\noindent
Manatanteraka ny asany araka ny feon'ny fieritreretany ary ao anatin'ny fanajàna
ny fitsipika fototra voafaritra araka ny fomba voadidy arahina ao amin'ny
andininy faha-79 etsy ambany ny solombavambahoaka.

\andininy{}Mandritra ny fe-potoana iasany, ny solombavambahoaka dia tsy afaka
miova vondrona politika mba hiditra any aminà vondrona vaovao, ankoatra ny
anaran'ny vondrona izay nahavoafidy azy, raha tsy izany dia aongana.\\

\noindent
Raha misy ny fandikàna ny andàlana etsy aloha, dia ambaran'ny Fitsarana Avo
momba ny Lalàmpanorenana ny sazy fanonganana.\\

\noindent
Ny solombavambahoaka voafidy ka tsy avy aminà antoko iray dia afaka miditra ao
amin'ny vondrona parlemantera araka ny safidiny eo anivon'ny Antenimiera.\\

\noindent
Ny fananganana Solombavambahoaka iray dia azon'ny Fitsarana Avo momba ny
Lalàmpanorenana anaovana fanambarana ihany koa, raha toa izy ka mivoana amin'ny
làlam-pitondran-tenan'ny vondrona parlemantera misy azy.\\

\noindent
Ny fomba fanonganana sy ny fitsipika momba ny toetra fototra ary ny fitandroana
ny hasin'ny asa dia faritana amin'ny alàlan'ny lalàna mikasika ny antoko
politika sy ny didy amam-pitsipika mikasika ny famatsiam-bola ny antoko
politika.

\andininy{}Tsy misy solombavambahoaka na dia iray aza azo enjehina, karohina,
samborina, tànana am-ponja, na tsaraina noho ny hevitra naposany na latsabato
nataony teo am-panaovana ny asany.\\

\noindent
Tsy misy Solombavambahoaka na dia iray aza azo enjehina sy samborina noho ny
heloka bevava na heloka tsotra mandritra ny fotoam-pivoriana raha tsy
nahazoan-dàlana tamin'ny Antenimiera, raha tsy hoe tratra ambodiomby.\\

\noindent
Tsy misy solombavambahoaka na dia iray aza azo samborina ivelan'ny
fotoam-pivoriana, raha tsy nahazoan-dàlana tamin'ny Biraon'ny Antenimiera, raha
tsy hoe tratra ambodiomby, noho ny fanenjehana nomena alàlana na ny fanamelohana
tanteraka.\\

\noindent
Afaka manao fitarainana an-tsoratra any amin'ny Birao maharitry ny
Antenimierampirenena izay rehetra olona manana fanamarihana momba ny tsy
fahavitan'ny solombavambahoaka iray ny asa nankinina taminy. Tsy maintsy manome
valiny amin'ny antsipiriany mikasika ilay raharaha ny Birao ao anatin'ny telo
volana.

\andininy{}Fidina eo amin'ny fiandohan'ny fotoam-pivoriana voalohany ny Filoha
sy ny mpikambana ao amin'ny Birao mandritry ny fotoam-piasany.\\

\noindent
Na izany aza noho ny antony lehibe, dia azo esorina amin'ny toerana misy azy
ireo avy ny mpikambana ao amin'ny Birao amin'ny alàlan'ny latsabato miafina
ataon'ny roa ampahatelon'ny solombavambahoaka.

\andininy{}Mivory ara-potoana indroa isan-taona ny Antenimierampirenena. Ny
faharetan'ny fotoam-pivoriana isanisany dia Ferana ho enimpolo andro.\\

\noindent
Manomboka ny talata voalohany amin'ny volana mey ny fotoam-pivoriana voalohany
ary ny talata fahatelo amin'ny volana oktobra kosa ny fotoam-pivoriana faharoa
izay atokana indrindra amin'ny fandaniana ny lalàna mifehy ny
fitantanam-bolam-panjakana.

\andininy{}Mivory tsy ara-potoana araka ny lahadinika voafaritra ny
Antenimierampirenena noho ny fitarihan'ny Praiminisitra, na noho ny
fangatahan'ny antsasa-manilan'ny mpikambana ao amin'ny Antenimierampirenena
araka ny didim-panjakana noraisin'ny Filohan'ny Repoblika teo amin'ny
Filankevitry ny Minisitra.\\

\noindent
Tsy mihoatra ny roa ambin'ny folo andro ny faharetan'ny fotoam-pivoriana. Na
izany aza anefa, raha vantany vao tapitra ny lahadinika izay niantsoana ny
Antenimieram-pirenena, dia faranana amin'ny alàlan'ny didim-panjakana ny
fivoriana.

\andininy{}Ampahibemaso ny fivorian'ny Antenimierampirenena.  Tanana
an-tsoratra izany, ary havoaka ho fantatry ny besinimaro araka ny fepetra
voalazan'ny lalàna.\\

\noindent
Mivory tsy ampahibemaso ny Antenimierampirenena araka ny fangatahana ataon'ny
ampahaefatry ny mpikambana ao aminy na ny fangatahana ataon'ny
Governemanta. Anaovana fitànana an-tsoratra ny fanapahana nofaranana.

\andininy{}Manao fivoriana manokana avy hatrany ny Antenimierampirenena ny
talata faharoa manaraka ny fanambarana ny vokatry ny fifidianana azy mba
hananganana ny biraony sy ny fanamboarana ny vaomiera.\\

\noindent
Manana zo amin'ny toeran'ny filoha lefitra iray ny mpanohitra ary mitarika ny
iray amin'ireo vaomiera farafahakeliny.\\

\noindent
Mifarana ny fivoriana raha vao tapitra ny lahadinika.

\andininy{}Ny fitsipika mikasika ny fampandehanana ny Antenimierampirenena dia
faritan'ny lalàna fehizoro ao amin'ny foto-kevitra ankapobe sy faritan'ny
fitsipika anatiny ao amin'ny fombafomba arahiny. Avoaka amin'ny
Gazetim-panjakan'ny Repoblika ny fitsipika anatiny.

\subsubsection{Ny amin'ny Antenimierandoholona}
\label{sec:ny-aminny-anten-1}

\andininy{}« Loholon'i Madagasikara » no anarana entin'ireo mpikambana ao
amin'ny Antenimierandoholona. Dimy taona no fe-potoana iasany, afa-tsy amin'izay
mikasika ny Filohan'ny Antenimierandoholona, ho fampiharana ny andininy faha-46
andalana faharoa amin'ity Lalàmpanorenana ity.

\andininy{}Ny Antenimierandoholona no misolo tena ireo Vondrom-bahoaka
Itsinjaram-pahefana sy ireo vondron-draharaha ara-toekarena ary
ara-tsosialy. Izany dia ahitana, ho an'ny roa ampahatelony, mpikambana voafidy
mitovy isa ho an'ny Faritany tsirairay, ny iray ampahatelony dia mpikambana
tendren'ny Filohan'ny Repoblika, ny ampahany araka ny fanolorana ataon'ny
vondrona izay mahasolo-tena be indrindra avy amin'ny hery ara-toekarena sy
ara-tsosialy ary ara-kolontsaina, ny ampahany noho ny fahaizana manokana.

\andininy{}Lalàna fehizoro no mamaritra ny fitsipika momba ny
fampandehanan-draharaha sy ny firafitry ny Antenimierandoholona.  Torak'izany
koa ny fomba fifidianana sy fanendrena ny mpikambana ao aminy.

\andininy{}Maka ny hevitry ny Antenimierandoholona ny Governemanta momba ny
raharaha ara-toekarena sy ara-tsosialy ary ny fandaminana ny Vondrom-bahoaka
Itsinjaram-pahefana.

\andininy{}Mivory ara-potoana indroa isan-taona ny Antenimierandoholona.
Enimpolo andro no faharetan'ny fotoam-pivoriana.\\

\noindent
Manomboka ny talata voalohany amin'ny volana mey ny fotoam-pivoriana voalohany
ary ny talata fahatelo amin'ny volana oktobra kosa ny fotoam-pivoriana faharoa
izay atokana indrindra amin'ny fandaniana ny lalàna mifehy ny
fitantanam-bolam-panjakana.\\

\noindent
Afaka manao fotoam-pivoriana manokana ihany koa izy araka ny fiantsoana ataon'ny
Governemanta. Ny lahadinika amin'izany dia ferana amin'ny alàlan'ny
didim-panjakana fiantsoana izay noraisana teo amin'ny Filankevitry ny Minisitra.\\

\noindent
Rehefa tsy mivory ny Antenimierampirenena, ny Antenimierandoholona dia tsy afaka
mandinika afa-tsy ny raharaha nangatahin'ny Governemanta ny heviny, afa-tsy izay
volavolan-dalàna.

\andininy{}Ny fepetra voalazan'ny andininy faha-71 hatramin'ny faha-79 dia
ampiharina amin'ny Antenimierandoholona ihany koa.

\subsubsection{Ny amin'ny fifandraisan'ny governemanta sy ny antenimiera}
\label{sec:ny-aminny-fifandr}

\andininy{}Samy manam-pahefana hanolotra volavolan-dalàna ny Praiminisitra, ny
Solombavambahoaka ary ny Loholona.\\

\noindent
Dinihina eo amin'ny Filankevitry ny Minisitra ny volavolan-dalàna ary apetraka
amin'ny birao iray amin'ireo biraon'ny Antenimiera roa.\\

\noindent
Ny lahadiniky ireo Antenimiera dia ahitana araka ny laharampahamehana sy araka
ny lahadinika faritan'ny Governemanta, ny ady hevitra momba ny volavolan-dalàna
napetraky ny Praiminisitra teo amin'ny Biraon'ny Antenimierampirenena na teo
amin'ny Biraon'ny Antenimierandoholona.\\

\noindent
Ny tolo-dalàna sy ny fanitsiana napetraky ny mpikambana ao amin'ny Antenimiera
dia ampahafantarina ny Governemanta, izay manana fe-potoana telopolo andro ho
an'ny tolo-dalàna ary dimy ambin'ny folo andro amin'ny fanitsiana mba hanaovany
fanamarihana.\\

\noindent
Rehefa tapitra izany fe-potoana izany, dia miroso amin'ny fandinihana ny
tolo-dalàna sy ny fanitsiana mba handaniana azy ny Antenimiera nametrahana
izany.\\

\noindent
Tsy azo raisina ny tolo-dalàna na ny fanitsiana raha toa ny fandaniana azy ireo
ka misy fiatraikany eo amin'ny sehatry ny fanatanterahana ny tetibola
an-dalam-panatanterahana, na ny fihenan'ny loharam-bolam-panjakana na ny
fitambesaran'ny lolohan'ny Fanjakana, afa-tsy izay mikasika ny lalàna mifehy ny
fitantanam-bolam-panjakana.\\

\noindent
Raha hita mandritra ny fomba arahina amin'ny fanaovan-dalàna fa tsy tafiditra ao
anatin'ny faritra sahanin'ny lalàna ny tolo-dalàna sy ny fanitsiana dia afaka
manohitra ny tsy fahazoana mandray izany ny Governemanta. Raha misy ny tsy
fifanarahana eo amin'ny Governemanta sy ny Antenimierampirenena na ny
Antenimierandoholona, dia mamoaka didy ao anatin'ny valo andro ny Fitsarana Avo
momba ny Lalàmpanorenana araka ny fangatahana ataon'ny Praiminisitra na ny
Filohan'ny iray amin'ireo Antenimiera mpanao lalàna.\\

\noindent
Fotoam-pivoriana herinandro roa amin'ny efatra farafahakeliny no atokana
handinihina ireo rijan-teny sy ireo adi-hevitra izay nangatahin'ny Governemanta
ho soratana eo amin'ny lahadinika.

\andininy{}Ny Antenimiera no mandany ny lalàna fehizoro, ny lalàna mifehy ny
fitantanam-bola ary ny lalàna tsotra araka ny fepetra faritan'ity
Lalàmpanorenana ity.

\andininy{}Ankoatra ireo anton-javatra ampisahanan'ny andininy hafa ao amin'ny
Lalàmpanorenana azy, dia lalàna fehizoro ihany no mametra~:

\begin{enumerate}
\item ny fitsipika mikasika ny fifidianana ny Filohan'ny Repoblika~;

\item ny fombafomba amin'ny fandatsaham-bato mikasika ny fifidianana
  solombavambahoaka, ny fepetra ahafahana milatsaka hofidina, ny fomba itondrana
  ny tsy fampirafesana ny asa aman-draharaha sy ny fanonganana, ny fitsipika
  momba ny fanoloana raha misy fahabangan-toerana, ny fandaminana ary ny
  fampandehanana ny Antenimierampirenena~;

\item ny fombafomba amin'ny fandatsaham-bato mikasika ny fifidianana Loholona,
  ny fepetra ahafahana milatsaka hofidina, ny fomba itondrana ny tsy
  fampirafesana ny asa aman-draharaha sy ny fanonganana, ny fitsipika momba ny
  fanoloana raha misy fahabangan-toerana, ny fandaminana ary fampandehanana ny
  Antenimierandoholona~;

\item Ny fitsipika mifehy ny fahefana, ny fomba fandaminana sy ny fampandehanana
  ny Vondrom-bahoakam-paritra Itsinjaram-pahefana ary koa ny fitsipika mifehy ny
  fitantanana ny raharahany manokana;

\item ny fandaminana, ny firafitra, ny fampandehanan-draharaha ary ny
  andraikitry ny Fitsarana Tampony sy ireo Fitsaran-dehibe telo mandrafitra
  azy. Toa izany koa ny mikasika ny fanendrena ireo mpikambana ao aminy sy ireo
  izay mikasika ny paika arahina eo anatrehany;

\item ny sata mifehy ny Mpitsara;

\item ny fandaminana sy ny fampandehanana ary ny anjara raharahan'ny
  Filankevitra Ambony momba ny Mpitsara~;

\item ny fandaminana, ny fampandehanana, ny anjara raharaha, ny
  fampakaran-draharaha ary ny paika ady arahina manoloana ny Fitsarana Avo~;

\item ny fandaminana, ny fampandehanana, ny anjara raharaha, ny
  fampakaran-draharaha ary ny paika arahina manoloana ny Fitsarana Avo momba ny
  Lalàmpanorenana~;

\item ny Fehezan-dalàna momba ny fifidianana~;

\item ny fepetra ankapobe mikasika ny lalàna mifehy ny
  fitantanam-bolam-panjakana~;

\item ny fepetra ankapobe mikasika ny Fifampiraharaham-barotra amin'ny Fanjakana
  amin'ny harena an-kibon'ny tany~;

\item ny fisian'ny toe-javatra mampihotakotaka ary koa ny famerana ny
  fahafahan'ny vahoaka, ny isam-batan'olona ary iombonana mandritra izany
  fotoana izany~;

\item ny fepetra fampitoviana ara-bola hisian'ny tombom-pitoviana eo amin'ny
  samy vondrom-bahoakam-paritra.
\end{enumerate}


\andininy{}Ny lalàna fehizoro dia laniana sy ovàna araka ireto fepetra
manaraka ireto~:
\begin{enumerate}
\item ny volavolan-dalàna na tolo-dalàna dia tsy aroso hotapahina sy holanian'ny
  Antenimiera voalohany nandray azy raha tsy tapitra ny fe-potoana dimy ambin'ny
  folo andro aorian'ny nametrahana azy~;

\item ny paika arahina voalazan'ny andininy faha-86, faha-96 ary faha-98 no
  ampiharina. Na izany aza anefa, tsy maintsy ny antsasa-manilan'ny mpikambana
  ao amin'ny Antenimiera tsirairay no mandany ny lalàna fehizoro, raha misy ny
  tsy fifanarahan-kevitra eo amin'ny Antenimiera roa tonta izay samy efa
  nandinika izany indroa, dia ny Antenimierampirenena amin'ny alàlan'ny
  latsabato iadanian'ny roa ampahatelon'ny mpikambana ao aminy no mandray
  fanapahan-kevitra farany.

  Raha toa ka tsy nolanian'ny Antenimierampirenena ny volavolan-dalàna fehizoro
  alohan'ny fiafaran'ny fotoam-pivoriana dia azo ampiharina amin'ny alàlan'ny
  hitsivolana ny fepetra entin'io volavolan-dalàna io, ka ampidirina ao raha
  ilaina izany, ny iray na maromaro amin'ireo fanitsiana nolanian'ny Antenimiera
  iray.

\item tsy maintsy mitovy ny fenitra andanian'ny Antenimiera roa tonta ny lalàna
  fehizoro mikasika ny Antenimierandoholona.
\end{enumerate}
\noindent
Tsy azo avoaka hanan-kery ny lalàna fehizoro raha tsy efa nambaran'ny Fitsarana
Avo momba ny Lalàmpanorenana fa mifanaraka amin'ny Lalàmpanorenana.

\andininy{}Ao anatin'ny faritry ny lalàna fehizoro ampiharina mikasika izany, ny
lalàna mifehy ny fitantanam-bolam-panjakana dia~:

\begin{enumerate}
\item mamaritra ny loharanom-bola sy ny lolohan'ny Fanjakana araka ny fepetra sy
  araka ny voalazan'ny lalàna fehizoro~;

\item mamaritra ho an'ny taom-piasana ny karazana, ny habetsahana, ary ny
  fanokanana ny loharanom-bola sy ny lolohan'ny Fanjakana ary koa ny
  fifandanjana eo amin'ny teti-bola sy ny vola vokatr'izany araka ny fanerena eo
  amin' ny lafiny toe-karena faobe.


\item mamaritra ny tahan'ny fidiram-bola tokony hiverina amin'ny Fanjakana sy ny
  Vondrom-bahoaka Itsinjaram-pahefana ary koa ny karazana sy ny tahan'ny hetra
  sy ny haba ambony indrindra raisina mivantana ho an'ny tetibolan'izany
  vondrom-bahoaka izany faritana eo amin'ny Filankevitry ny Minisitra.
\end{enumerate}
\noindent
Ny lalàna fehizoro no mamaritra ny fomba fampiharana ireo fepetra voalazan'ity
andininy ity ary koa ny fepetra momba ny fampitoviana ny vola hisian'ny
tombom-pitoviana eo amin'ny samy Vondrom-bahoaka Itsinjaram-pahefana.\\

\noindent
Ny lalàna no mamaritra mazava ny fepetra fisamboram-bola sy manapaka ny mety ho
fananganana ny tahiry.\\

\noindent
Ny lalàna no mamaritra~:

\begin{itemize}
\item Ny fombafomba fampiasana ny tahiry fisamboram-bola avy any ivelany sy ny
  fanaraha-maso ataon'ny Antenimiera sy araka ny lalàna;
  
\item Ny fomba fitondrana ny andraikitry ny tena manokana sy ara-bolan'ny
  fahefana momba ny fitantanam-bola, tompo-mariky ny fanodinkodinana ny vola
  nosamborina ary koa ny tsy fidiran'ny Fanjakana andraikitra amin'izany.
\end{itemize}

\andininy{}Ny lalàna momba ny fandaharanasa no mamaritra ny tanjona kendrena
amin'ny asa ataon'ny Fanjakana mikasika ny toekarena, tontolo iainana, sosialy
ary fanajariana ny tany.\\

\noindent
Ny fepetra voalazan'ity andininy ity dia soritana mazava sy fenoina amin'ny
alàlan'ny lalàna fehizoro.

\andininy{}Mandritra ny fivoriana ara-potoana faharoa no andinihan'ny
Antenimiera ny volavolan-dalàna mifehy ny fitantanam-bolam-panjakana.\\

\noindent
Ny Minisitra miandraikitra ny Fitantanam-bola sy ny Tetibola no manomana ny
volavolan-dalàna mifehy ny fitantanam-bolam-panjakana ka eo ambany fahefan'ny
Praiminisitra, Lehiben'ny Governemanta no anaovany izany.\\

\noindent
Manana fe-potoana enimpolo andro farafahabetsany ny Antenimiera handinihana
izany.\\

\noindent
Manana fe-potoana telopolo andro farahahabetsany manomboka eo amin'ny fotoana
nanolorana ny volavolan-dalàna taminy ny Antenimierampirenena mba handinihana
azy am-boalohany. Rehefa tsy naneho ny heviny tao anatin'izay fe-potoana izay
izy, dia heverina ho toy ny efa nandany azy sahady ka dia atolotra ny
Antenimierandoholona indray ilay volavolan-dalàna.\\

\noindent
Mitovy amin'izay fepetra izay ihany, dia manana fe-potoana dimy ambin'ny folo
andro manomboka ny fampitana ny volavolan-dalàna ny Antenimierandoholona mba
handinihany azy am-boalohany ary ny Antenimera tsirairay dia manana fe-potoana
dimy andro hanaovany ny fandinihana azy manaraka.\\

\noindent
Raha misy iray amin'ireo Antenimiera tsy maneho ny heviny tao anatin'ny
fe-potoana voatondro, dia heverina ho toy ny efa nanao latsabato nanekena ilay
rijan-teny natolotra ho dinihana izy.\\

\noindent
Raha tsy nolanian'ny Antenimiera alohan'ny fiafaran'ny fotoam-pivoriana faharoa
ny volavolan-dalàna momba ny fitantanam-bola, dia ampidirina ao ny fanitsiana
iray na maromaro neken'ny Antenimiera roa tonta.\\

\noindent
Izay fanitsiana rehetra natao tamin'ilay volavolan-dalàna momba ny tetibola ka
miteraka fitomboan'ny fandaniana na fihenan'ny loharanom-bolam-panjakana dia tsy
maintsy ampiarahina amin'ny tolo-kevitra momba ny fampitomboana ny vola miditra
na ny toekarena mifandanja amin'izany.\\

\noindent
Raha tsy napetraka ara-potoana mba holaniana alohan'ny fiantombohan'ny
taom-piasana ny volavolan-dalàna momba ny fitantanam-bolam-panjakana amin'ny
taom-piasana iray, dia omena alàlana ny Praiminisitra ahazoany manao ny
famoriana ny hetra ary manokatra amin'ny alàlan'ny didim-panjakana ny sorabola
mifandraika amin'ireo asa nolaniana tamin'ny latsabato.\\

\noindent
Lalàna fehizoro no mamaritra ny fomba fankatoavana ny volavolan-dalàna momba ny
fitantanam-bolam-panjakana.

\andininy{}Ny Fitsarana momba ny Kaonty no manampy ny Antenimiera amin'ny
fanaraha-maso ny asa ataon'ny Governemanta. Manampy ny Antenimiera sy ny
Governemanta amin'ny fanaraha-maso ny fanatanterahana ny lalàna mifehy ny
fitantanam-bolam-panjakana ary koa amin'ny fanombanana ny politikam-panjakana
izy. Amin'ny fifandraisany amin'ny vahoaka, izy dia mandray anjara amin'ny
fampahafantarana ny olom-pirenena.\\

\noindent
Ny kaontin'ny Fitondran-draharaham-panjakana dia tokony ho ara-dalàna sady
marina.  Manome endrika azo itokiana amin'ny vokatry ny fitantanany, ny fananany
ary ny toe-bola izany.

\andininy{}Ny Filohan'ny Repoblika no mifandray amin'ny Antenimiera amin'ny
alàlan'ny hafatra tsy azo iadian-kevitra.

\andininy{}Ankoatra ireo raharaha ampisahanin'ny andininy hafa ao amin'ny
Lalàmpanorenana azy dia~:

\begin{enumerate}[I.]
\item Ny lalàna no mametra ny fitsipika mikasika~:
  \begin{enumerate}[1.]
  \item ny zon'ny olom-pirenena sy ny antoka fototra nomena ny tsirairay,
    fikambanana, antoko politika ary ny vondrona hafa rehetra ho amin'ny
    fampiasana ny zo sy ny fahalalahana ary koa ny adidy aman'andraikitr'izy
    ireo~;

  \item ny fifandraisana iraisam-pirenena~;

  \item ny zom-pirenena~;

  \item ny Banky Foibe sy ny fomba ny famoaham-bola~;

  \item ny fivezivezen'ny olona~;

  \item ny fitsipika mikasika ny paika arahina amin'ny ady madio sy ny
    raharaham-barotra~;

  \item ny fitsipika mikasika ny paika arahina amin'ny ady atao amin'ny
    Fanjakana sy ny paika arahina amin'ny fitantanam-bolam-panjakana~;

  \item ny famaritana ny heloka bevava sy ny heloka tsotra ary koa ny sazy
    ampiharina amin'izany, ny paika ady heloka, ny famotsoran-keloka~;

  \item ny fitsipika mikasika ny fifanoheran'ny lalàna sy ny an'ny fahefana~;

  \item ny fananganana rafi-pitsarana vaovao sy ny fahefana tandrify azy avy ary
    koa ny fandaminana sy ny fitsipika mifehy ny paika arahina izay ampiharina
    amin'izany~;

  \item ny fandaminana ny fianakaviana, ny sata sy ny fizakan-jon'ny
    isam-batan'olona, ny fehin'ny fanambadiana, ny fandovana sy fanomezana~;

  \item ny fifehezana araka ny lalàna ny fitompoana, ny fizakà-manana, ny adidy
    ateraky ny zo isam-batan'olona sy ny ara-barotra ary ny fepetra ahazoana
    manaisotra amin'ny tompony ny fananany na ny fampiasana azy tsy azo lavina
    noho ny filan'ny besinimaro na ny famindrana ny fitompoana azy amin'ny
    Fanjakana~;

  \item ny fananganana ny sokajin'antokon-draharaham-panjakana~;

  \item ny sata sy ny fitondrana ny fizakan-tenan'ireo Oniversite, ary koa ny
    sata mifehy ny mpampianatra eny amin'ny fampianarana ambaratonga ambony~;

  \item ny sori-dàlana lehibe amin'ny fanomezan-danja ny fampianarana
    ambaratonga fototra sy faharoa~;

  \item ny loharanon-karena iankinan'ny fiainam-pirenena~;

  \item ny fandaminana sy ny fampandehanana ny Vondrombahoakam-paritra
    itsinjaram-pahefana~;

  \item ny sata manokan'ny Renivohitry ny Repoblika, ny ampahany sasantsasany
    amin'ny tanim-pirenena, ireo lapam-panjakana sy ireo trano isan'ny
    fananam-panjakana, ireo seranana sy ny zotram-pifamoivoizana, ireo
    seranam-piaramanidina, ary koa ny satan'ireo harena an-dranomasina~;

  \item ny karazana, ny fototra ary ny taha farany ambony amin'ny hetra sy haba
    any amin'ny Vondrombahoakam-paritra itsinjaram-pahefana~;

  \item ny Filankevitry ny Andrin'ny Mari-boninahitra Malagasy~;

  \item ny fanatsarana ny tanàna sy ny fonenana~;

  \item ny fepetra amin'ny fisitrahan'ny vahiny tany~;

  \item ny fepetra famindrana amin'ny Fanjakana ny tany tsy nohamaintisamolaly~;

  \item ny fandaminana ny fampandehanana ary anjara raharahan'ny
    Fisafoan-draharaham-panjakana Ankapobe ary ireo rantsana hafa momba ny
    fanaraha-maso ny Fitondran-draharaham-panjakana.

  \end{enumerate}

\item Ny lalàna no mamaritra ny foto-kevitra ankapobe momba~:
  \begin{enumerate}[1.]

  \item ny fandaminana ny fiarovam-pirenena sy ny fampiasan'ireo manam-pahefana
    sivily ny Foloalindahy na ny Hery mpitandro ny filaminana~;

  \item ny sata ankapobe mifehy ny mpiasam-panjakana sivily sy miaramila eto
    amin'ny Firenena ary ny mpiasam-panjakana any amin'ny faritra~;

  \item ny lalàna mifehy ny asa, ny zo sendikaly, ny zo hitokona ary ny
    fitsinjovana ara-tsosialy~;

  \item ny famindrana ny fananan'ny orinasa na antokon-draharaha miankina
    amin'ny Fanjakana amin'olon-tsotra na ny mifamadika amin'izany~;

  \item ny fandaminana na ny fampandehanana ny sehatrasa samihafa misahana
    lalàna, toekarena, sosialy ary kolontsaina~;

  \item ny fiarovana ny tontolo iainana.
 
  \end{enumerate}
\item Ny fivorian'ny Antenimiera roa tonta mitambatra ho Kongresy irery, amin'ny
  alàlan'ny latsabato iandanian'ny antsasa-manilan'ny mpikambana rehetra ao
  anatiny, no afaka manome alàlana hamakiana ady amin'ny firenen-kafa.
\end{enumerate}


\andininy{}Ny volavolan-dalàna sy ny tolo-dalàna rehetra dia dinihin'ny
Antenimiera nandray azy voalohany vao ampitaina any amin'ny Antenimiera
ankilany. Ifandimbiasan'ny Antenimiera roa tonta ny ady hevitra mandra-pisian'ny
fandaniana rijan-teny tokana.\\

\noindent
Raha toa ka tsy mety lany ny volavolan-dalàna na tolo-dalàna iray rehefa
nodinihin'ny Antenimiera tsirairay indroa ka misy tsy fifanarahan-kevitra
tamin'izy roa tonta, na rehefa nodinihin'ny Antenimiera tsirairay indray mandeha
rehefa nanambara ny fisian'ny hamehana ny Governemanta, dia afaka mandray
fanapahan-kevitra ny Praiminisitra hamory vaomiera ikambanana isasahana izay
ampiandraiketina fanolorana rijan-teny momba ireo fepetra mbola
iadian-kevitra. Ny rijan-teny novolavolan'io vaomiera ikambanana io dia azon'ny
Governemanta atolotra ny Antenimiera roa tonta mba hankatoavina. Tsy misy
fanitsiana azo raisina raha tsy nahazoana ny faneken'ny Governemanta.\\

\noindent
Raha toa ka tsy afaka mandany rijan-teny tokana iombonana ny vaomiera na tsy
nolaniana araka ireo fepetra voalaza ao amin'ny andalana etsy aloha ilay
rijan-teny, dia ny Antenimierampirenena, amin'ny alàlan'ny latsabato
iandanian'ny antsasa-manilan'ny mpikambana ao aminy no manapa-kevitra farany.

\andininy{}Manana endrika didy amam-pitsipika ny raharaha hafa tsy tafiditra ao
anatin'ny faritra sahanin'ny lalàna. Azo ovàna amin'ny alàlan'ny didim-panjakana
izay avoaka rehefa milaza ny heviny ny Fitsarana Avo momba ny Lalàmpanorenana
ireo rijan-teny miendrika lalàna tafiditra ao anatin'izany raharaha izany.\\

\noindent
Ireo rijan-teny miendri-dalàna mety ho raisina, rehefa manan-kery ity
Lalàmpanorenana ity dia tsy azo ovàna amin'ny alàlan'ny didim-panjakana raha tsy
nambaran'ny Fitsarana Avo momba ny Lalàmpanorenana fa miendrika didy
amam-pitsipika izany araka ny voalazan'ny andalana etsy aloha.

\andininy{}Ny Governemanta, raha manamby ny andraikitra araka ny fepetra
voalazan'ny andininy faha-100 etsy ambany, dia afaka mitaky amin'ny Antenimiera
tsirairay mba hanapa-kevitra avy amin'ny alàlan'ny latsabato tokana mikasika ny
fepetra manontolo na ampahany voalazan'ny rijan-teny iadian-kevitra~:

\begin{itemize}
\item mandritra ny fivoriana tsy ara-potoana, raha toa ka voatolotra valo amby
  efapolo ora mialoha ny fisokafan'ny fivoriana ireo rijan-teny ireo~;

\item ao anatin'ny valo andro farany amin'ny fotoam-pivoriana ara-potoana
  tsirairay avy.
\end{itemize}

\andininy{}Ao anatin'ny telopolo andro nanendrena azy no anoloran'ny
Praiminisitra ny fandaharan'asany momba ny fampiharana ny politika ankapoben'ny
Fanjakana amin'ny Antenimiera izay afaka manome tolo-kevitra.\\

\noindent
Raha tsapan'ny Governemanta eo am-panatanterahana ny asa fa misy fanovàna
fototra tokony hatao amin'io fandaharanasa io, dia entin'ny Praiminisitra eny
amin'ny Antenimierampirenena izay afaka manome tolo-kevitra izany fanovàna
izany.

\andininy{}Aorian'ny fandinihana nataon'ny Filankevitry ny Minisitra, dia
azon'ny Praiminisitra atao ny manamby ny andraikitry ny Governemantany amin'ny
fametrahana fangataham-pitokisana.\\

\noindent
Tsy azo atao ny fandatsa-bato raha tsy afaka valo amby efapolo ora aorian'ny
fametrahana ny fangataham-pitokisana.  Raha toa ka ny antsasa-manilan'ny
mpikambana ao amin'ny Antenimierampirenena no mitsipaka ny fangataham-pitokisana
dia mametraka ny fialany amin'ny Filohan'ny Repoblika ny Governemanta.\\

\noindent
Ny Filohan'ny Repoblika no manendry ny Praiminisitra araka ny andininy faha- 54.

\andininy{}Isaky ny fiandohan'ny fivoriana ara-potoana voalohany no
anoloran'ny Governemanta amin'ny Antenimierampirenena ny tatitra momba ny
fanatanterahana ny fandaharanasany.\\

\noindent
Arahina adihevitra mahakasika ny asan'ny Governemanta sy ny fanombanana ny
politikam-panjakana ny fanolorana.

\andininy{}Ny fomba ampiasain'ny Antenimiera ahafantarany ny asa sahanin'ny
Governemanta dia ny fanontaniana am-bava, ny fanontaniana an-tsoratra, ny
fanadinana hentitra ary ny fampiasana ny vaomiera mpanao famotorana.\\

\noindent
Fivoriana iray isaky ny dimy ambin'ny folo andro farafahakeliny, ka anatin'izany
ny fivoriana atao mandritra ny fotoam-pivoriana tsy ara-potoana voalazan'ny
andininy faha-76 dia atokana ho an'ny fanontanian'ireo mpikambana ao amin'ny
Antenimiera sy ireo valiny avy amin'ny Governemanta.\\

\noindent
Fivoriana telo andro isam-bolana no atokana ho amin'ny lahadinika izay
notapahin'ny Antenimiera tsirairay araka ny fitarihan'ny vondron'ny mpanohitra
ao amin'ny Antenimiera voakasika ary koa avy ny an'ny vodron'ireo vitsy an'isa.

\andininy{}Azon'ny Antenimierampirenena atao ny maneho tsy fankasitrahana ny
andraikitry ny Governemanta amin'ny alàlan'ny latsabato fitsipaham-pitokisana.\\

\noindent
Tsy azo raisina anefa ny tolo-kevitra toy izany raha tsy ny antsasaky ny
mpikambana ao amin'ny Antenimierampirenena no manao sonia azy. Valo amby efapolo
ora aorian'ny fametrahana ny tolo-kevitra hotapahina vao azo atao ny latsabato.\\

\noindent
Tsy lany ny fitsipaham-pitokisana raha tsy ny roa ampahatelon'ny mpikambana ao
amin'ny Antenimierampirenena no mankato azy.\\

\noindent
Raha lany izany fitsipaham-pitokisana izany dia mametraka ny fialany amin'ny
Filohan'ny Repoblika ny Governemanta.

\andininy{}Ny Antenimiera amin'ny alalan'ny latsa-bato ataon'ny
antsasa-manilan'ny mpikambana mandrafitra ny Antenimiera tsirairay dia afaka
mamindra ny fahefany amin'ny Filohan'ny Repoblika hanao lalàna mandritra ny
fotoana voafetra sy mikasika toe-javatra voafaritra.\\

\noindent
Ny fanomezam-pahefana dia manome alàlana ny Filohan'ny Repoblika handray fepetra
amin'ny ankapobeny momba ny toe-javatra mahakasika ny sehatry ny lalàna, amin'ny
alalan'ny hitsivolana raisina eo amin'ny Filankevitry ny Minisitra.

\subsection{Ny amin'ny filankevitra ara-toekarena, ara-tsosialy ary ara-kolontsaina}
\label{sec:ny-aminny-filank}

\andininy{}Ny Filankevitra ara-toekarena, ara-tsosialy ary ara-kolontsaina
izay ampakaran'ny Governemanta raharaha, no manome ny heviny mikasika ny
volavolan-dalàna, hitsivolana na didim-panjakana ary koa mikasika ireo
tolo-dalàna izay naroso hodinihiny.\\

\noindent
Manam-pahefana izy handinika ny volavolà sy tolo-dalàna manana endrika
ar-toekarena ara-tsosialy ary ara-kolontsaina ankoatra ny lalàna momba ny
fitantanam-bolam-panjakana. Afaka mandray amin'ny alalan'ny tenany manokana ireo
fandinihana na famotopotorana rehetra mikasika ireo raharaha ara-toekarena,
ara-tsosialy ary ara-kolontsaina izy. Ampitaina any amin'ny Filohan'ny Repoblika
ny tatitra ataony.\\

\noindent
Ny firafitra, ny anjara raharaha ary ny fampandehanana ny Filan-kevitra
ara-toekarena, ara-tsosialy ary ara-kolontsaina dia Ferana amin'ny alalan'ny
lalàna fehizoro.

\subsection{Ny amin'ny fitsarana}
\label{sec:ny-aminny-fitsarana}

\subsubsection{Ny amin'ny fe-kevitra fototra}
\label{sec:ny-aminny-fe}

\andininy{}Eto amin'ny Repoblikan'i Madagasikara, ny Fitsarana Tampony, ny
Fitsarana Ambony sy ireo fitsarana miankina aminy ary koa ny Fitsarana Avo no
mitsara araka ny Lalàmpanorenana sy araka ny lalàna, amin'ny anaran'ny Vahoaka
Malagasy.

\andininy{}Ny Filohan'ny Repoblika no miantoka ny fahaleovantenan'ny
fitsarana.\\

\noindent
Amin'izany izy, dia ampian'ny Filankevitra Ambony momba ny Mpitsara izay izy no
Filohany.\\

\noindent
Ny Minisitra misahana ny Fitsarana no Filoha lefitra.\\

\noindent
Ny Filankevitra Ambony momba ny Mpitsara, rantsa-mangaika miandraikitra ny
fiarovana, ny fitantanana ny fiainan'ny mpiasa manao ho anton-draharaha ny asa
sy ny famaizana ny mpitsara no miandraikitra ny~:
\begin{itemize}
\item fitandroana indrindra indrindra ny fanajana ny lalàna sy ireo fepetra
  raketin'ny sata mifehy ny Mpitsara,

\item fanaraha-maso ny fanajana ny fitsipika momba ny fitandroana ny hasin'ny
  asa amin'ny maha mpitsara,

\item fanomezana torolàlana hanatsarana ny fomba
  fitantanan-draharaham-pitsarana, indrindra amin'izay mikasika ny fepetra eo
  amin'ny lafin'ny lalàna na didy amam-pitsipika mikasika ny antokom-pitsarana
  sy ny Mpitsara.
\end{itemize}

\noindent
Ny mambra ao amin'ny Governemanta, ny Antenimiera, ny Filankevitra Ambony momba
ny fiarovana ny demokrasia sy ny Fanjakana tan-dalàna, ny Lehiben'ny Fitsarana
ary koa ny fikambanana natsangana ara-dalàna dia afaka mitondra ny raharaha
manoloana ny Filankevitra Ambony momba ny Mpitsara.\\

\noindent
Ny fitsipika mikasika ny fandaminana, ny fampandehanana ary ny anjara
raharahan'ny Filankevitra Ambony momba ny Mpitsara dia ferana amin'ny alàlan'ny
lalàna fehizoro.

\andininy{}Eo amin'ny asam-pitsarana sahaniny dia mahaleotena ny Mpitsara
mpamoaka didy, ny mpitsara sy mpitsara mpanampy ka ny Lalàmpanorenana sy ny
lalàna ihany no mifehy azy ireo.\\

\noindent
Amin'izany, ankoatra ny toe-javatra voalazan'ny lalàna ary na dia eo aza ny
fahefana ara-pitsipi-pifehezana, dia tsy azo tohintohinina na amin'ny fomba
inona na amin'ny fomba inona izy ireo amin'ny asam-pitsarana ataony ho
fisahanana ny andraikiny

\andininy{}Ny mpitsara mpamoaka didy dia tsy azo hetsehina amin'ny toerany~;
manao ny asa mifanentana amin'ny toerana tokony hisy azy araka ny laharany~; tsy
azo afindra toerana raha tsy misy fanekena avy aminy afa-tsy hoe misy antony
ilàna izany amin'ny asa, voamarin'ny Filankevitra Ambony momba ny Mpitsara
ara-dalàna.

\andininy{}Ny Mpitsara ao amin'ny Fampanoavana dia fehezin'ny
ambaratongam-pahefana~; na izany aza anefa, ao amin'ny fehin-teny na fitakiana
am-bava ataony, dia izay feon'ny fieritreretany sy araka ny lalàna no mibaiko
azy. Mampiasa ireo mpiandraikitra ny fikarohana fandikan-dalàna izay ahafahany
manara-maso ny asa sy ny fampandehanan-draharaha izy.\\

\noindent
Ny fangatahana azy ireo hanatanteraka zavatra hita miharihary fa mifanohitra
amin'ny lalàna dia mitarika fanasaziana araka ny lalàna ho an'ireo mpampanao
izany.

\andininy{}Ny fanaovana ny asa maha Mpitsara dia tsy azo ampirafesina amin'ny
asa sahanina eo anivonà antoko politika sy eo anivon'ny Governemanta, amin'ny
asa fanatanterahana andraikitry ny olom-boafidy eo anivon'ny vahoaka na izay
rehetra mety ho fisahanana asa aman-draharaha ahazoam-bola, afa-tsy ny asa
fampianarana.\\

\noindent
Ny mpitsara amperinasa rehetra dia tsy mahazo maka fiandaniana ara-politika.\\

\noindent
Ny mpitsara misahana andraikitra maha olom-boafidy dia heverina ho toy ny
nafindra hanao asa hafa.

\andininy{}Ny Fisafoan-draharaha Ankapoben'ny Fitsarana, ahitana ny
solontenan'ny Antenimiera, ny solontenan'ny Governemanta ary ny solontenan'ny
Filankevitra Ambony momba ny fiarovana ny demokrasia sy ny Fanjakana tan-dalàna
ary ny solontenan'ny Mpitsara no miandraikitra ny fanaraha-maso ny fanajana ny
fitsipika manokana momba ny fitandroana ny hasin'ny asa amin'ny maha Mpitsara,
ary koa ny fihetsika ataon'ny mpiasan'ny fitsarana.\\

\noindent
Miankina amin'ny Fiadidiana ny Repoblika izy.\\

\noindent
Ny Filohan'ny Repoblika, ny Antenimiera, ny Governemanta, ny Lehiben'ny
Fitsarana, ny fikambanana natsangana ara-dalàna ary ny olona rehetra manana
tombontsoa amin'izany dia afaka mitondra fitarainana manoloana ny
Fisafoan-draharaha ankapobe ny Fitsarana.\\

\noindent
Ny fitsipika mikasika ny fandaminana, ny fampandehanana ary ny anjara
raharahan'ny Fisafoan-draharaha ankapobe ny Fitsarana dia ferana amin'ny
alàlan'ny lalàna.

\andininy{}Ny Filan-kevi-pirenena momba ny Fitsarana, rantsa-mangaika
fakan-kevitra ahitana ny Filoha Voalohany ao amin'ny Fitsarana Tampony, Filoha,
ny Tonia Voalohany Mpampanoa ao amin'ny Fitsarana Tampony, sy ireo Lehiben'ny
Fitsarana Ambony sy Fitsarana Tampony, ny solontenan'ny fahefana mpanatanteraka,
ny fahefana mpanao lalàna, ny Fitsarana Avo momba ny Lalàmpanorenana, ny
Filankevitra Ambony momba ny Mpitsara, ny Filankevitra Ambony momba ny fiarovana
ny demokrasia sy ny Fanjakana tan-dalàna ary ny mpiasan'ny fitsarana amin'ny
ankapobeny. Amin'izany, izy dia afaka manolotra amin'ny Governemanta ny fepetra
eo amin'ny lafiny lalàna na didy amam-pitsipika mikasika ny fandaminana sy
fampandehanana ny antokom-pitsarana, ny sata mifehy ny Mpitsara sy ny mpiasan'ny
fitsarana.\\

\noindent
Ny fitsipika mikasika ny fandaminana, ny fampandehanana ary ny anjara
raharahan'ny Filankevi-pirenena momba ny Fitsarana dia ferana amin'ny alàlan'ny
lalàna.

\subsubsection{Ny amin'ny fitsarana avo momba ny lalàmpanorenana}
\label{sec:ny-aminny-fitsarana-1}

\andininy{}Misy mpikambana sivy ao amin'ny Fitsarana Avo momba ny
Lalàmpanorenana. Ny fe-potoana iasany dia fito taona tsy azo havaozina.\\

\noindent
Mpikambana telo amin'izy ireo no tendren'ny Filohan'ny Repoblika, roa fidin'ny
Antenimierampirenena, roa fidin'ny Antenimierandoholona, ary roa fidin'ny
Filankevitra Ambony momba ny Mpitsara.\\

\noindent
Ny Filohan'ny Fitsarana Avo momba ny Lalàmpanorenana dia fidin'ny mpikambana
amin'ireo mpikambana ao amin'io Fitsarana io.\\

\noindent
Izany fifidianana izany ary koa ny fanendrena ny mpikambana hafa dia toavina
amin'ny alàlan'ny didim-panjakana raisin'ny Filohan'ny Repoblika.

\andininy{}Ny asa maha mpikambana ao amin'ny Fitsarana Avo momba ny
Lalàmpanorenana dia tsy azo ampirafesina amin'ny asan'ny mpikambana ao amin'ny
Governemanta, ao amin'ny Antenimiera, amin'izay rehetra andraikitra maha
olom-boafidim-bahoaka eo anivon'ny vahoaka, amin'izay rehetra asa aman-draharaha
hafa ahazoam-bola afa-tsy ny asa fampianarana, ary koa ny asa aman-draharaha
rehetra eo anivon'ny antoko politika na eo anivon'ny sendikà.\\

\noindent
\andininy{}Ankoatra ny raharaha ampisahanin'ny andininy hafa ao amin'ny
Lalàmpanorenana, ary araka ny fepetra voalazan'ny lalàna fehizoro dia ny
Fitsarana Avo momba ny Lalàmpanorenana no~:

\begin{enumerate}
\item mitsara raha mifanaraka amin'ny Lalàmpanorenana ireo fifanekena, lalàna,
  hitsivolana, ary ireo didy amam-pitsipika mahaleotena~;

\item mandamina ny fifanolanana ara-pahefana eo amin'ny Andrim-panjakana roa na
  maromaro, na eo amin'ny Fanjakana sy ny Vondrom-bahoaka Itsinjaram-pahefana
  iray na maromaro na eo amin'ny Vondrom-bahoaka Itsinjaram-pahefana roa na
  maromaro~;

\item mitsara raha mifanaraka amin'ny Lalàmpanorenana ny fanapahana sy didy
  amam-pitsipika nankatoavin'ny Vondrom-bahoaka Itsinjaram-pahefana~;

\item mitsara ny raharaha mikasika ny fizotry ny fitsapan-kevi-bahoaka, ny
  fifidianana ny Filohan'ny Repoblika ary ny fifidianana solombavambahoaka sy ny
  loholona~;

\item manambara ny vokatra ofisialin'ny fifidianana ny Filohan'ny Repoblika,
  solombavambahoaka ary ny fakana ny hevitra amin'ny alàlan'ny
  fitsapan-kevi-bahoaka.
\end{enumerate}

\andininy{}Ny Filohan'ny Repoblika dia mandroso ny lalàna fehizoro sy ny lalàna
ary ny hitsivolana hodinihin'ny Fitsarana Avo momba ny Lalàmpanorenana izay
manapaka raha mifanaraka amin'ny Lalàmpanorenana, alohan'ny amoahana azy ireo
hanan-kery.\\

\noindent
Ny fepetra nambara fa tsy mifanaraka amin'ny Lalàmpanorenana dia tsy azo avoaka
hanan-kery. Raha izany no miseho, ny Filohan'ny Repoblika dia afaka manapaka, na
hamoaka hanan-kery ireo fepetra hafa amin'ilay lalàna na ilay hitsivolana, na
mandroso ny rijan-teny manontolo amin'ny fanapahana ataon'ny Antenimiera na ny
Filankevitry ny Minisitra indray, araka ny trangan-javatra, na koa tsy hamoaka
azy hanan-kery.\\

\noindent
Amin'ireo trangan-javatra voalaza etsy ambony ireo, ny fampakaran-draharaha eo
amin'ny Fitsarana Avo momba ny Lalàmpanorenana dia mampihantona ny fe-potoana
tokony hamoahana ny lalàna hanan-kery. Ny fitsipika anatin'ny Antenimiera
tsirairay dia aroso hamarinina raha mifanaraka amin'ny Lalàmpanorenana alohan'ny
hampiharana azy. Ny fepetra nambara fa tsy mifanaraka amin'ny Lalàmpanorenana
dia tsy azo ampiharina.

\andininy{}Ny Lehiben'ny Andrim-panjakana na ny ampahaefatry ny mpikambana ao
amin'ny iray amin'ireo Antenimiera roa tonta na ny rantsa-mangaika ao amin'ny
Vondrom-bahoakam-paritra Itsinjaram-pahefana na ny Filankevitra Ambony momba ny
fiarovana ny demokrasia sy ny Fanjakana tan-dalàna dia afaka mandroso rijan-teny
miendrika lalàna na fitsipika, ary koa izay rehetra anton-javatra tafiditra
amin'ny fahefan'ny Fisarana Avo momba ny Lalàmpanorenana mba hohamarinina ny
fifanarahany amin'ny Lalàmpanorenana.\\

\noindent
Raha misy mpiady manasingana fifanoherana amin'ny Lalàmpanorenana eo
anatrehan'ny Fitsarana, dia mampihantona ny fizotry ny raharaha io fitsarana io
ary mampakatra ny raharaha eo amin'ny Fitsarana Avo momba ny Lalàmpanorenana ao
anatin'ny iray volana.\\

\noindent
Torak'izany raha misy mpiady milaza eo anatrehan'ny fitsarana fa misy fepetra ao
amin'ny rijan-tenin-dalàna na fitsipika manohintohina ny zo fototra ananany
eken'ny Lalàmpanorenana dia mampiato ny fizotry ny raharaha io fitsarana io
araka ny fepetra voalazan'ny andalana etsy aloha.\\

\noindent
Ny fepetra nambara fa tsy mifanaraka amin'ny Lalàmpanorenana dia mitsahatra tsy
manan-kery avy hatrany.\\

\noindent
Avoaka amin'ny Gazetim-panjakana ny fanapahana noraisin'ny Fitsarana Avo momba
ny Lalàmpanorenana.

\andininy{}Azon'ny Lehiben'ny Andrim-panjakana sy ny rantsa-mangaika ao amin'ny
Vondrom-bahoakam-paritra Itsinjaram-pahefana atao ny maka ny hevitry ny
Fitsarana Avo momba ny Lalàmpanorenana, mikasika ny fifanarahana na tsia amin'ny
Lalàmpanorenana ny volavolam-panapahan-kevitra na ny fivoasana ny fepetra iray
amin'ity Lalàmpanorenana ity.

\andininy{}Mamoaka didim-pitsarana ny Fitsarana Avo momba ny Lalàmpanorenana
amin'ny fifanolanana miseho amin'ny fifidianana sy fakàna ny hevitry ny vahoaka
mivantana.\\

\noindent
Manao fanapahana kosa izy amin'ny anton-javatra hafa tandrifin'ny fahefana
ananany, afa-tsy ny trangan-javatra voalazan'ny andininy faha-119.\\

\noindent
Lazaina ny anton'ny didim-pitsarana sy fanapahana avoakan'ny Fitsarana Avo momba
ny Lalàmpanorenana ary tsy azo akarina fitsarana hafa intsony. Tsy maintsy
ampiharina amin'ny fahefam-panjakana ary koa izay rehetra manam-pahefana amin'ny
fitondrana sy ny fitsarana izy ireny.

\subsubsection{Ny amin'ny fitsarana tampony}
\label{sec:ny-aminny-fitsarana-2}

\andininy{}Ny Fitsarana Tampony no manara-maso ny fampandehanana ara-dalàna ny
antokom-pitsarana misahana ny ady madio sy ady heloka, ny ady atao amin'ny
Fanjakana ary ny fitantanam-bolam-panjakana.

Toy izao no firafiny~:
\begin{itemize}
\item ny Fitsarana Fandravana;

\item ny Filankevi-panjakana;

\item ny Fitsarana momba ny Kaonty
\end{itemize}

\andininy{}Ny Filoha Voalohany sy ny Tonia Voalohany Mpampanoa ao amin'ny
Fitsarana Tampony no lehiben'io antokom-pitsarana avo io.\\

\noindent
Izy ireo dia tendrena avy, amin'ny alàlan'ny didim-panjakana noraisina teo
amin'ny Filankevitry ny Minisitra araka ny tolokevitra nataon'ny Filankevitra
Ambony momba ny Mpitsara.  Raha azo atao dia avy amin'ireo mpitsara tranainy
indrindra ao amin'ny laharana ambony indrindra avy amin'ny antokom-pitsarana
misahana ady madio sy ady heloka, ady atao amin'ny Fanjakana ary
fitantanam-bolam-panjakana no tendrena.

\andininy{}Misy Filoha lefitra telo manampy ny Filoha Voalohany ao amin'ny
Fitsarana Tampony ka ampiandraiketin-draharaha tsirairay avy ao amin'ny
fiadidiana ny Fitsarana Fandravana, ny Filankevi-panjakana ary ny Fitsarana
momba ny Kaonty.\\

\noindent
Ny Filoha lefitra tsirairay dia tendrena eo amin'ny Filankevitry ny Minisitra
amin'ny alàlan'ny didim-panjakana avy amin'ny Filohan'ny Repoblika araka ny
tolokevitra nataon'ny Filankevitra Ambony momba ny Mpitsara. Raha azo atao avy
amin'ireo mpitsara tranainy indrindra ao amin'ny laharana ambony indrindra ao
amin'ny antokom-pitsarana misahana ady madio sy ady heloka, ady atao amin'ny
Fanjakana ary fitantanam-bolam-panjakana no tendrena.

\andininy{}Ny Fampanoavana ao amin'ny Fitsarana Tampony dia ahitana~:

\begin{itemize}
\item ny Fampanoavana ao amin'ny Fitsarana fandravana~;

\item ny Kaomisaria jeneralin'ny lalàna ho an'ny Filan-kevi-panjakana~;

\item ny Kaomisaria jeneraly momba ny Tahirim-bolam-panjakana ho an'ny Fitsarana
  momba ny Kaonty.
\end{itemize}

Ny lehiben'ireo antokon-draharaha telo ireo no manampy ny Tonia Voalohany
Mpampanoa ao amin'ny Fitsarana Tampony.
\\

\noindent
Ny lehiben'ny Fampanoavana ao amin'ny Fitsarana Fandravana, ny Kaomisaria
jeneralin'ny lalàna na ny Kaomisaria jeneraly momba ny Tahirim-bolam-panjakana
dia tendrena eo amin'ny Filankevitry ny Minisitra araka ny tolokevitra nataon'ny
Filankevitra Ambony momba ny Mpitsara~; raha azo atao dia avy amin'ireo mpitsara
tranainy indrindra ao amin'ny laharana ambony indrindra ao amin'ny
antokom-pitsarana misahana ady madio sy ady heloka, ady atao amin'ny Fanjakana
ary fitantanam-bolam-panjakana no tendrena.

\andininy{}Ankoatra ireo anjara raharaha ampisahanin'ny lalàna manokana azy
dia ny Fitsarana Tampony no mitsara ny fifanolanana mikasika ny fahefa-mitsara
eo amin'ny fitsarana roa avy amin'ny antokom-pitsarana samihafa.

\andininy{}Ny Fitsarana Fandravana no manara-maso ny fampiharan'ireo
antokom-pitsarana misahana ny ady madio sy ady heloka ny lalàna.\\

\noindent
Ankoatra ireo fahefana eken'ny lalàna manokana hosahaniny dia izy no manapaka ny
amin'ny fangatahana fandravana atao amin'ny didy navoakan'ireo antoko-pitsarana
izay tsy misy intsony fampakarana azo atao aminy.

\andininy{}Ankoatran'ny fahefana sy fepetra manokana voalazan'ny lalàna ny
Filankevi-panjakana no manamarina ny maha-ara-dalàna ny fanapahana ataon'ny
Fitondran-draharaham-panjakana ary mitandro ny fampiharana ny lalàna ataon'ny
antoko-pitsarana misahana ny ady atao amin'ny Fanjakana.\\

\noindent
Ny Filankevi-panjakana, araka ny fepetra faritan'ny lalàna fehizoro no~:

\begin{enumerate}
\item mitsara ny fangatahana fanafoanana ny fanapahana noraisin'ny
  manam-pahefana ao amin'ny fitondran-draharaha foibe, ny fangatahana
  famerenan-jo noho ny fahavoazana vokatry ny asan'ny
  Fitondran-draharaham-panjakana, ny fitarainana amin'ny fifanolanana mikasika
  ny hetra~;

\item mandinika ny fampakarana fitsarana ambony ny fanamarinana ny maha
  ara-dalàna ny fanapahana nataon'ny manam-pahefana ao amin'ny
  Vondrom-bahoakam-paritra Itsinjaram-pahefana~;

\item mitsara ny fampakarana fitsarana ambony na ny fandravana ireo didy
  navoakan'ny fitsarana misahana ny ady atao amin'ny Fanjakana na
  antokom-pitsarana misahana ny ady atao amin'ny Fanjakana manana fahefa-mitsara
  manokana.
\end{enumerate}
\noindent
Izy no mitsara ireo raharaha sasantsasany ifanolanana amin'ny fifidianana.\\

\noindent
Azon'ny Praiminisitra sy ny mambra ao amin'ny Governemanta atao ny maka hevitra
aminy mba hanomezany ny heviny, mikasika ny volavolan'ny rijan-tenin-dalàna sy
ny didy amam-pitsipika ary ny fifanarahana na momba ny fivoasana ny voalazan'ny
lalàna sy ny didy amam-pitsipika.\\

\noindent
Araka ny fangatahan'ny Praiminisitra dia afaka mandinika rijan-tenin-dalàna
mikasika ny fandaminana, ny fomba fiasa sy ny andraikitry ny
sampan-draharaham-panjakana izy.

\andininy{}Ny Fitsarana momba ny Kaonty no manara-maso ny fampiharan'ireo
antokom-pitsarana misahana ny fitantanam-bola ny lalàna.  Ankoatra ireo fahefana
eken'ny lalàna manokana hosahaniny dia izy no~:

\begin{enumerate}
\item mitsara ny kaontin'ireo mpitàm-bolam-panjakana~;

\item manara-maso ny fanatanterahana ny lalàna mifehy ny
  fitantanam-bolam-panjakana sy ny tetibolan'ny antokon-draharaha miankina
  amin'ny Fanjakana~;

\item manara-maso ny kaonty sy ny fitantanana ny orinasam-panjakana~;

\item mitsara ny fampakarana ambony natao tamin'ireo didim-pitsarana navoakan'ny
  fitsarana momba ny fitantanam-bolam-panjakana na ireo
  antokon-draharaham-panjakana miendrika fitsarana~;

\item manampy ny Antenimiera sy ny Governemanta eo amin'ny fanaraha-maso ny
  fanatanterahana ny lalàna mifehy ny fitantanam-bolam-panjakana.
\end{enumerate}

\andininy{}Ny Fitsarana Tampony dia mandefa tatitra isan-taona momba ny asa
nosahaniny any amin'ny Filohan'ny Repoblika, ny Praiminisitra, ireo Filohan'ny
Antenimiera roa tonta sy ny Minisitra miandraikitra ny Fitsarana ary ny
Filankevitra Ambony momba ny Mpitsara.\\

\noindent
Io tatitra io dia tsy maintsy avoaka amin'ny Gazetim-panjakana amin'ny taona
manaraka ny fikatonan'ny taom-pitsarana voakasika.

\andininy{}Ny Filoha Voalohany, ny Tonia Mpampanoa ao amin'ny Fitsarana
Ambony dia tendrena eo amin'ny Filankevitry ny Minisitra amin'ny alàlan'ny
didim-panjakana noraisin'ny Filohan'ny Repoblika araka ny tolo-kevitry ny
Filankevitra Ambony momba ny Mpitsara. Raha azo atao, avy amin'ireo mpitsara
tranainy indrindra ao amin'ny laharana ambony indrindra ao amin'ny
antokom-pitsarana misahana ady madio sy ady heloka, ady atao amin'ny Fanjakana
ary fitantanam-bolam-panjakana no tendrena.

\subsubsection{Ny amin'ny fitsarana avo}
\label{sec:ny-aminny-fitsarana-3}

\andininy{}Ny Filohan'ny Repoblika dia tsy tompon'andraikitra noho ny
zava-natao teo amin'ny asany na teo am-panaovana ny raharahany afa-tsy ny
amin'ny famadihana Tanindrazana, ny fandikana bevava na ny fandikana
miverimberina ny Lalàmpanorenana, na koa ny fanaovana antsirambina ny andraikiny
ka hita miharihary fa tsy mifanaraka amin'ny fisahanana ny asa maha-filoha.\\

\noindent
Ny Antenimierampirenena no afaka miampanga azy amin'ny alàlan'ny latsabato atao
ampahibemaso ary lanian'ny roa ampahatelon'ny mpikambana ao aminy.\\

\noindent
Azo tsaraina eo anoloan'ny Fitsarana Avo Izy. Ny fiampangana dia mety hiafara
amin'ny fanalàna azy amin'ny toerany.

\andininy{}Raha misy ny fanonganana ny Filohan'ny Repoblika dia ny Fitsarana
Avo momba ny Lalàmpanorenana no mizaha fototra ny fahabangan'ny toerany ary
tanterahina araka ny fepetra voalazan'ny andininy faha-47 etsy ambony ny
fifidianana Filoha vaovao. Ny Filoha voaongana dia tsy mahazo milatsaka hofidina
intsony amin'izay rehetra mety ho asan'olom-boafidim-bahoaka.

\andininy{}Ny Filohan'Antenimiera, ny Praiminisitra, ny mpikambana hafa ao
amin'ny Governemanta ary ny Filohan'ny Fitsarana Avo momba ny Lalàmpanorenana
dia tompon'andraikitra itataovam-pamaizana noho ny zava-natao teo amin'ny asany
ka voatondro ho heloka bevava na heloka tsotra tamin'ny fotoana nanaovana izany,
ka ny Fitsarana Avo no mitsara azy.\\

\noindent
Ny Antenimierampirenena no afaka miampanga azy ireo amin'ny amin'ny alàlan'ny
latsabato atao ampahibemaso lanian'ny antsasa-manilan'ny mpikambana.\\

\noindent
Ny Tonia Voalohany Mpampanoa ao amin'ny Fitsarana Tampony no manao ny
fanenjehana.

\andininy{}Ny Filohan'Antenimiera, ny Praiminisitra, ny mpikambana hafa ao
amin'ny Governemanta ary ny Filohan'ny Fitsarana Avo momba ny Lalàmpanorenana
dia azon'ny antoko-pitsarana momba ny lalàna ifampitondran'ny daholobe tsaraina
raha mandika lalàna tamin'ny fotoana tsy anatanterahany andraiki-panjakana.\\

\noindent
Ny Tonian'ny Fampanoavan'ny Fitsarana Fandravana no manao ny fanenjehana.\\

\noindent
Amin'izany raha misy ny heloka tsotra, ny antokom-pitsarana famaizana mahefa dia
tarihin'ny Filohan'ny Fitsarana, ary raha misy ny tsy fahafahany dia filoha
lefitra iray no mitarika izany.\\

\noindent
Ampiharina koa amin'ireo solombavambahoaka, loholona ary ny mpikambana ao
amin'ny Fitsarana Avo momba ny Lalàmpanorenana ny fepetra voalazan'ireo andàlana
telo etsy aloha ireo.

\andininy{}Manana fahefana feno sy tanteraka hitsara ny Fitsarana Avo.

\andininy{}Misy mpikambana iraika ambinifolo ao amin'ny Fitsarana Avo. Ireto avy
izy ireo~:
\begin{enumerate}
\item ny Filoha Voalohan'ny Fitsarana Tampony, Filoha izay soloin'ny Filohan'ny
  Fitsarana Fandravana avy hatrany raha misy tsy fahafahany~;

\item Filohan'ny Rantsana roa avy ao amin'ny Fitsarana Fandravana sy mpisolo
  toerana roa tendren'ny Fivoriam-ben'io Fitsarana io~;

\item Filoha Voalohany roa avy amin'ny Fitsarana Ambony sy mpisolo toerana roa
  tendren'ny Filoha Voalohan'ny Fitsarana Tampony~;

\item Solombavambahoaka tompon-toerana roa sy solombavambahoaka mpisolo toerana
  roa nofidin'ny Antenimierampirenena amin'ny fiantombohan'ny fe-potoana
  fanaovan-dalàna~;

\item Loholona tompon-toerana roa sy loholona mpisolo toerana roa nofidin'ny
  Antenimierandoholona amin'ny fiantombohan'ny fe-potoana fanaovan-dalàna.

\item Mpikambana roa tompon-toerana sy mpikambana roa mpisolo toerana avy
  amin'ny Filankevitra Ambony momba ny fiarovana ny demokrasia sy ny Fanjakana
  tan-dalàna.
\end{enumerate}
\noindent
Ny Tonia Voalohany Mpampanoa ao amin'ny Fitsarana Tampony ampian'ny iray na
maromaro amin'ireo mpikambana ao amin'ny fampanoavana iadidiany no misolo tena
ny fampanoavana. Raha misy tsy fahafahan'ny Tonia Voalohany Mpampanoa dia ny
Tonia Mpampanoa ao amin'ny Fitsarana Fandravana no misolo toerana azy.\\

\noindent
Ny lehiben'ny Firaketan-draharahan'ny Fitsarana Tampony no mpiraki-draharahan'ny
Fitsarana Avo.  Izy no mitazona ny firaketana an-tsoratra. Raha misy tsy
fahafahany dia ny lehiben'ny firaketan-draharahan'ny Fitsarana Fandravana no
misolo azy.\\

\noindent
Ny fandaminana sy ny paika arahina manoloana ny Fitsarana Avo dia feran'ny
lalàna fehizoro.


\section{Ny amin'ny fifanekena sy fifanarahana iraisam-pirenena}
\label{sec:ny-aminny-fifanekena}

\andininy{}Ny Filohan'ny Repoblika no mifampiraharaha sy mankato ny fifanekena
iraisam-pirenena.  Ampahafantarina azy izay fifampiraharahana rehetra mahakasika
fifanekena iraisam-pirenena tsy ilàna fankatoavana.\\

\noindent
Tsy maintsy lalàna no manome alàlana amin'ny fankatoavana na fanekena ny
fifanekem-pihavanana, ny fifanekem-barotra, ny fifanekena na fifanarahana
mikasika ny fandaminana iraisam-pirenena, ireo izay mampiditra andraikitra ho
hefaina amin'ny volam-panjakana tafiditra amin'izany ny fisamboram-bola avy any
ivelany, ireo izay manova ny fepetra miendrika lalàna, ireo izay mikasika ny
fitoetry ny olona, fifanekena tsy hifanafika, ireo izay manova ny faritry ny
tanim-pirenena.\\

\noindent
Alohan'izay fankatoavana rehetra atao, dia atolotry ny Filohan'ny Repoblika ny
Fitsarana Avo momba ny Lalàmpanorenana ireo fifanekena, mba hohamarinina raha
mifanaraka amin'ny Lalàmpanorenana.\\

\noindent
Raha misy tsy fifanarahana amin'ny Lalàmpanorenana, dia tsy azo atao ny
fankatoavana raha tsy aorian'ny fanovàna ny Lalampanorenana.\\

\noindent
Raha vantany vao navoaka ho fantatry ny besinimaro ny fifanekena na ny
fifanarahana nankatoavina na nekena ara-dalàna dia manan-kery mihoatra noho ny
lalàna raha toa ampiharin'ny ankilany ny fifanarahana na ny fifanekena
tsirairay.\\

\noindent
Ny fifanekena mahakasika firotsahanan'i Madagasikara amina Vondrom-paritra dia
tsy maintsy aroso amin'ny fitsapa-kevi-bahoaka hakàna ny heviny.

\andininy{}Ny Praiminisitra no manao ny fifampiraharahana sy manao sonia ny
fifanarahana iraisam-pirenena tsy ilàna fankatoavana.

\section{Ny amin'ny fandaminana ny fiadidiana ny lafin-tany}
\label{sec:ny-aminny-fand-1}

\subsection{Ny amin'ny fepetra ankapobe}
\label{sec:ny-aminny-fepetra}


\andininy{}Ireo Vondrombahoakam-paritra Itsinjaram-pahefana, manana zo
aman'andraikitra eken'ny lalàna ary mizaka tena ara-pitantanana sy ara-bola, no
sehatra eo amin'ny rafi-pitondrana ahazoan'ny olom-pirenena mandray anjara
amin'ny fitantanana ny raharaham-bahoaka ary miantoka ny fanehoana izay
fahasamihafana misy eo amin'izy ireo sy ny mampiavaka azy.\\

\noindent
Lalàna no mamaritra ny fari-pananany izay ahitana ny fananana ampiasaina ho
amin'ny tombontsoam-bahoaka sy ny fananana manokana.\\

\noindent
Ny tany tsy hita tompo sy ireo tsy manan-tompo dia anisan'ny fananana
tantanan'ny Fanjakana.

\andininy{}Manana fahefana handray didy amam-pitsipika ny
Vondrombahoakam-paritra Itsinjaram-pahefana.\\

\noindent
Ny Fanjakana no mitandro ny tsy hisian'ny fifanipahan'ny didy amam-pitsipiky ny
Vondrombahoakam-paritra Itsinjaram-pahefana iray amin'ny tombontsoan'ny
Vondrombahoakam-paritra Itsinjaram-pahefana iray hafa.\\

\noindent
Ny Fanjakana no miahy ny hisian'ny fampandrosoana mirindra eo amin'ireo samy
Vondrombahoakam-p aritra Itsinjaram-pahefana ka izany dia mifototra amin'ny
firaisan-kinam-pirenena, ny harena azo trandrahana ananan'ny faritra tsirairay
avy ary ny fifandanjana isam-paritra amin'ny alàlan'ny fepetra fifampitsinjarana
mifanentana.\\

\noindent
Handraisana fepetra manokana ho ny fampandrosoana ireo faritra izay tratra
aoriana, anisan'izany ny famoronana tahiry manokana ho amin'ny firaisan-kina.

\andininy{}Ireo Vondrombahoakam-paritra Itsinjaram-pahefana ampian'ny Fanjakana
no mitandro indrindra ny filaminam-bahoaka, ny fiarovana ny fiaraha-monina, ny
fitantanan-draharaha, ny fanajariana ny tany, ny fampandrosoana ara-toekarena,
ny fiarovana ny tontolo iainana ary ny fanatsarana ny fari-piainana.\\

\noindent
Amin'ireo lafiny voalaza ireo dia ny lalàna no mametra ny fitsinjarana ny
fari-pahefana mba hitsinjovana ny tombontsoam-pirenena sy ny tombontsoan'ny
faritra.

\andininy{}Mizaka tena ara-bola ireo Vondrombahoakam-paritra
Itsinjaram-pahefana.\\

\noindent
Mamolavola sy mitantana ny tetibolany izy ireo, araka ny feni-kevitra ampiharina
amin'ny fitantanana ny volam-panjakana.\\

\noindent
Vatsiana amin'ny loharanom-bola maro samihafa ny tetibolan'ny
Vondrombahoakam-paritra Itsinjaram-pahefana.

\andininy{}Ny Kaominina, ny Faritra ary ny Faritany no
Vondrombahoakam-paritra Itsinjaram-pahefana eto amin'ny Repoblika.\\

\noindent
Ny famoronana sy ny famaritana ny Vondrombahoakam-paritra Itsinjaram-pahefana
dia tsy maintsy mifototra amin'ny fitoviana eo amin'ny toe-tany, toekarena,
sosialy ary kolontsaina. Ny lalàna no mamaritra izany.

\andininy{}Malalaka ny fitantanana tena malalaka ireo Vondrombahoakam-paritra
Itsinjaram-pahefana ka amin'ny alàlan'ny fanapahan-kevitry ny antenimierany no
andamina ny raharahany araka ny tandrifim-pahefana omen'ny Lalàmpanorenana sy ny
lalàna azy ireo.\\

\noindent
Tsy tokony mifanohitra amin'ny Lalàmpanorenana sy ny didy aman-dalàna izany
fanapahan-kevitra izany.

\andininy{}Ny fisoloantena ny Fanjakana eo anivon'ny Vondrombahoakam-paritra
Itsinjaram-pahefana dia fehezin'ny lalàna.

\andininy{}Ny Fanjakana dia miandraikitra ny fampiharana ireto fepetra manaraka
ireto~:
\begin{itemize}
\item fitsinjarana ny fari-pahefana eo amin'ny Fanjakana sy ny
  vondrombahoakam-paritra Itsinjaram-pahefana~;

\item fitsinjarana ny loharanom-bola eo amin'ny Fanjakana sy ny
  vondrombahoakam-paritra Itsinjaram-pahefana~;

\item fitsinjarana ny raharaham-panjakana eo amin'ny Fanjakana sy ny
  vondrombahoakam-paritra itsinjaram-pahefana.
\end{itemize}

\andininy{}Ny fidiram-bolan'ny Vondrombahoakam-paritra Itsinjaram-pahefana dia
ahitana indrindra~:

\begin{itemize}
\item Ny vola azo avy amin' ny hetra sy haba nolanian'ny Filankevitra ary
  takiana mivantana harotsaka amin'ny tetibolan'ny Vondrombahoakam-paritra
  Itsinjaram- pahefana~; lalàna no mamaritra ny karazana sy ny sanda ambony
  indrindra amin'ireny hetra sy haba ireny izay hijerena indrindra ny
  andraikitra iantsorohan'ny Vondrombahoakam-paritra Itsinjaram-pahefana sy ny
  loloha ara-ketra ankapobe mitambesatra amin'ny Firenena~;

\item Ny anjara tokony ho azy avy hatrany amin'ny vola azo amin'ny hetra sy ny
  haba miditra ao amin'ny tetibolan'ny Fanjakana~; izany anjara izany izay
  sintonina avy hatrany amin'ny fotoana famoriana ny vola dia faritan'ny lalàna
  araka ny isan-jato izay ijerena ny andraikitra sahanin'ny
  Vondrombahoakam-paritra Itsinjaram-pahefana amin'ny ankapobeny sy
  isam-batan'olona mba hiantohana ny fampandrosoana ara-toekarena sy
  ara-tsosialy mifandanja eo amin'ireo Vondrombahoakam-paritra
  Itsinjaram-pahefana rehetra manerana ny tanim-pirenena~;

\item ny vola vokatry ny fanampiana avy amin'ny Fanjakana, voatokana na tsia ho
  amin'ny asa, ary nomena avy amin'ny teti-bolam-panjakana ho an'ny
  Vondrombahoakam-paritra Itsinjaram-pahefana rehetra na ny iray amin'izy ireo
  ka atao mifandanja amin'ny toe-javatra manokana misy eo aminy, na
  hamenoan'ireo Vondrombahoakam-paritra Itsinjaram-pahefana ny adidiny ara-bola
  izay nateraky ny fandaharan'asa na tetik'asa notapahan'ny Fanjakana ary
  tanterahan'ireo Vondrombahoakam-paritra Itsinjaram-pahefana~;

\item ny vola azo avy amin'ny fanampiana avy any ivelany ka tsy averina sy ny
  vola azo avy amin'ny fanomezana nomena ny Vondrombahoakam-paritra
  Itsinjaram-pahefana~;

\item ny vola miditra avy amin'ny fananany manokana;

\item ny fisamboram-bola izay ferana araka ny lalàna ny fepetra momba izany.
\end{itemize}

\subsection{Ny amin'ny rafitra}
\label{sec:ny-aminny-rafitra}

\subsubsection{Ny amin'ny kaominina}
\label{sec:ny-aminny-kaominina}


\andininy{}Ny kaominina no Vondrombahoakam-paritra Itsinjaram-pahefana
fototra. Ny kaominina dia atao hoe ambonivohitra na ambanivohitra ka ny
fisokajiana dia miankina amin'ny tahan'ny mponina sy ny lafim-pivoarana ary ny
fiitaran'ny Tanàna.

\andininy{}Ny kaominina dia mandray anjara amin'ny fampandrosoana ara-toekarena,
ara-tsosialy, ara-kolotsaina sy momba ny tontolo iainana eo amin'ny
fari-piadidiam-paritra misy azy. Ny fari-pahefany dia mikasika indrindra ny
feni-kevitry ny Lalàmpanorenana sy ny lalàna ary koa ny feni-kevitra momba ny
fanatonana ny vahoaka, ny fampivoarana ary ny fiarovana ny tombontsoan'ny
mponina.

\andininy{}Ny kaominina dia afaka mivondrona ho amin'ny fanatanterahana
tetikasa fampandrosoana iombonana.

\andininy{}Ao amin'ny kaominina, ny asa fanatanterahana sy fanapahan-kevitra
dia sahanin'ny rantsa-mangaika miavaka sy nofidina tamin'ny latsa-bato
andraisan'ny rehetra anjara mivantana.\\

\noindent
Ny mpikambana, ny fandaminana, ny anjara raharaha ary ny fampandehanana ny
rantsa-mangaika mpanatanteraka sy mpanapa-kevitra ary koa ny fomba sy ny fepetra
momba ny fifidianana ny mpikambana ao amin'ny kaominina dia ferana araka ny
lalàna.

\andininy{}Ny Fokonolona, nalamina ho fokontany eo anivon'ny kaominina, no
fototry ny fampandrosoana sy ny firindrana eo amin'ny lafiny fiaraha-monina sy
ny kolontsaina ary ny tontolo iainana.\\

\noindent
Ny tompon'andraikitra ao amin'ny fokontany dia mandray anjara amin'ny
famolavolana ny fandaharanasa momba ny fampandrosoana ny kaominina ao aminy.

\subsubsection{Ny amin'ny faritra}
\label{sec:ny-aminny-faritra}

\andininy{}Miompana indrindra amin'ny lafiny toekarena sy fiaraha-monina ny
andraikitry ny faritra.\\

\noindent
Ny faritra no misahana ny fampiroborobona sy fandrindrana ny fampandrosoana
ara-toekarena sy ara-tsosialy ao anatin'ny fari-piadidiany ka mandrafitra ny
fomba fandaminana, ny fanajariana ny tany ary ny fampiharana ny hetsika
fampandrosoana rehetra miaraka amin'ny antokon-draharaha miankina na tsy
miankina amin'ny Fanjakana.

\andininy{}Ny asan'ny mpanatanteraka dia sahanin'ny rantsa-mangaika
tarihin'ny Lehiben'ny Faritra izay olo-boafidy tamin'ny alalan'ny latsabato
andraisan'ny rehetra anjara.\\

\noindent
Ny Lehiben'ny Faritra no tompon'andraikitra voalohany amin'ny tetika sy ny
fanatanterahana ny asa rehetra mikasika ny fampandrosoana ara-toekarena sy
ara-tsosialy ao amin'ny faritra misy azy.\\

\noindent
Izy no Lehiben'ny Fitondran-draharaham-panjakana eo amin'ny faritra.

\andininy{}Ny asa fanapahan-kevitra dia sahanin'ny Filankevi-paritra ka ny
mpikambana dia olom-boafidy tamin'ny latsabato andraisan'ny rehetra anjara.\\

\noindent
Ny solombavambahoaka sy ny loholona avy amin'ny fari-piadidiana samihafa ao
amin'ny faritra dia mpikambana avy hatrany ao amin'ny Filankevi-paritra, ary
anisany mpanapan-kevitra.

\andininy{}Ny mpikambana, ny fandaminana, ny anjara raharaha ary ny
fampandehanana ny rantsa-mangaika mpanatanteraka sy mpanapa-kevitra ary koa ny
fomba sy ny fepetra momba ny fifidianana ny mpikambana ao amin'ny faritra dia
ferana araka ny lalàna.

\subsubsection{Ny amin'ny faritany}
\label{sec:ny-aminny-faritany}

\andininy{}Ny Faritany dia vondrombahoakam-paritra itsinjaram-pahefana,
manana zo aman'andraikitra eken'ny lalàna ary koa mizaka-tena ara-pitondrana sy
ara-bola.\\

\noindent
Izy no miantoka ny fandrindrana sy ny fampirindrana ny asa fampandrosoana ho
an'ny tombontsoan'ny faritany ary miahy ny fampandrosoana ara-drariny sy
mirindra ireo Vondrombahoakam-paritra Itsinjaram-pahefana ao amin'ny faritany.\\

\noindent
Ny faritany no manatanteraka ny politika momba ny fampandrosoana izay nofaritana
sy notapahina teo amin'ny filankevi-paritany.\\

\noindent
Ny faritra no misahana ny fampiroborobona sy fandrindrana ny fampandrosoana
ara-toekarena sy ara-tsosialy ao anatin'ny fari-piadidiany ka mandrafitra ny
fomba fandaminana, ny fanajariana ny tany ary ny fampiharana ny hetsika
fampandrosoana rehetra miaraka amin'ny antokon-draharaha miankina na tsy
miankina amin'ny Fanjakana.

\andininy{}Ny asan'ny mpanatanteraka dia sahanin'ny rantsa-mangaika
tarihin'ny Lehiben'ny Faritany izay olom-boafidy tamin'ny alalan'ny latsabato.\\

\noindent
Ny Lehiben'ny Faritany no tompon'andraikitra voalohany amin'ny tetika sy ny
fanatanterahana ny asa rehetra mikasika ny fampandrosoana ara-toekarena sy
ara-tsosialy ao amin'ny faritany misy azy.\\

\noindent
Izy no Lehiben'ny Fitondran-draharaham-panjakana eo amin'ny faritany.

\andininy{}Ny asa fanapahan-kevitra dia sahanin'ny Filankevi-paritany ka ny
mpikambana ao aminy dia olom-boafidy amin'ny alàlan'ny latsabato andraisan'ny
rehetra anjara.\\

\noindent
Ny solombavambahoaka sy ny loholona avy amin'ny fari-piadidiana samihafa ao
amin'ny faritany dia mpikambana avy hatrany ao amin'ny Filankevi-paritany, ary
anisany mpanapan-kevitra.

\andininy{}Ny ho anisany, ny fandaminana, ny anjara raharaha ary ny
fampandehanana ny rantsa-mangaika mpanatanteraka sy mpanapa-kevitra ary koa ny
fomba sy ny fepetra momba ny fifidianana ny mpikambana ao aminy dia ferana araka
ny lalàna.

\section{Ny amin'ny fanitsiana ny lalàmpanorenana}
\label{sec:ny-aminny-fanitsiana}

\andininy{}Tsy misy fanitsiana azo atao amin'ny lalampanorenana na dia iray
aza raha tsy hita fa tena ilaina

\andininy{}Raha toa ka hita fa tena ilaina ny fanitsiana dia anjaran'ny
Filohan'ny Repoblika no manapaka eo amin'ny Filankevitry ny Minisitra, na ny
Antenimiera amin'ny alalan'ny latsabato misaraka izay nolanian'ny roa
ampahatelon'ny mpikambana no manao izany.\\

\noindent
Ny volavola na tolokevitra fanitsiana dia tsy maintsy ankatoavin'ny telo
ampahaefatry ny mpikambana ao amin'ny Antenimierampirenena sy ny
Antenimierandoholona.\\

\noindent
Ny volavola na tolokevitra izay nankatoavina dia aroso amin'ny
fitsapan-kevi-bahoaka

\andininy{}Ny maha-Repoblika ny Fanjakana, ny foto-kevitra maha-iray tsy
anombinana ny tanim-pirenena, ny foto-kevitry ny fisarahan'ny fahefana, ny
foto-kevitry ny fizakan-tenan'ny Vondrombahoakam-paritra Itsinjaram-pahefana, ny
faharetana sy ny isan'ny fe-potoana iasan'ny Filohan'ny Repoblika, dia tsy azo
anaovana fanitsiana.\\

\noindent
Ny fahefana manokana ananan'ny Filohan'ny Repoblika ao anatin'ny
fisehoan-javatra mampihotakotaka na savorovoro politika dia tsy manome azy ny zo
hanao fanitsiana ny Lalàmpanorenana.


\section{Fepetra tetezamita sy samihafa}
\label{sec:fepetra-tetez-sy}

\andininy{}Izao Lalàmpanorenana izao dia holaniana amin'ny alàlan'ny
fitsapan-kevi-bahoaka. Ampiharina izany raha vantany vao havoakan'ny Filohan'ny
Fahefana Avon'ny Tetezamita hanan-kery, ao anatin'ny folo andro manaraka ny
fanambaràna farany ny vokatry ny fitsapan-kevi-bahoaka ataon'ny Fitsarana Avo
momba ny Lalàmpanorenana.

\andininy{}Ny lalàna manan-kery dia mbola mihatra hatrany amin'ireo fepetra
rehetra izay tsy mifanohitra amin'ity Lalàmpanorenana ity.\\

\noindent
Ny rijan-tenin-dalàna mikasika ireo Andrim-panjakana sy ireo rantsa-mangaika ary
koa ny lalàna fampiharana voalazan'ity Lalampanorenana ity dia raisina amin'ny
alalan'ny hitsivolana.

\andininy{}Mandra-pametrahana amin'ny toerana miandàlana ny Andrim-panjakana
voalazan'izao Lalàmpanorenana izao, ireo Andrim-panjakana sy ireo
rantsa-mangaika voalaza ho amin'ny fe-potoanan'ny Tetezamita dia manohy misahana
ny asany.\\

\noindent
Mitsahatra avy hatrany ny asan'ny Filankevitra Ambonin'ny Tetezamita sy ny
Kongresin'ny tetezamita vantany vao voafidy ny biraon'ny Antenimiera vaovao.\\

\noindent
Eo am-piandrasana ny fametrahana amin'ny toerany ny Antenimierandoholona dia
mizaka ny fahefana rehetra hanao lalàna ny Antenimierampirenena.\\

\noindent
Mandrampahatongan'ny fotoana andraisan'ny Filohan'ny Repoblika vaovao ny
fahefany dia manohy misahana ny asan'ny Filoham-panjakana ny Filohan'ny Fahefana
Avon'ny Tetezamita.\\

\noindent
Raha misy fahabangan-toeran'ny Filoha, n'inon'inona antony dia iarahan'ny
Praiminisitra, Filohan'ny Filankevitra Ambony ny Tetezamita ary ny Filohan'ny
Kongresy misahana ny asan'ny Filoham-panjakana.

\andininy{}Ho fanajàna ny Lalampanorenana, ny Filohan'ny Repoblika dia miantso
ireo ambaratongam-pahefana mahefa mba hanolotra ireo ho mpikambana ao amin'ny
Fitsarana Avo ahafahana manangana io rafitra io. Izany dia tanterahina ao
anatin'ny 12 volana manaraka ny fotoana nandraisany fahefana, ka raha vao
tapitra io fe-potoana io dia atsangana avy hatrany ny Fitsaràna Avo. Raha misy
ny tsy fahatomombanana dia azon'ny rehetra mahaporofo tombontsoa atao ny
mampakatra ny raharaha manoloana izay andrim-panjakana mahefa mba ny hangatahana
fanasaziana.\\

\noindent
Raha misy antony manokana, amin'izay mikasika ny Filohan'ny Repoblika, ny
Fitsarana Avo momba ny Lalàmpanorenana dia omem-pahefana handray ny sazy izay
mety ho noraisin'ny Fitsarana Avo raha toa izy ka efa nijoro.

\andininy{}Ao anatin'ny tondrozotra mahakasika ny fampihavanam-pirenena, dia
atsangana ny Filankevitry ny Fampihavanana Malagasy; ka ny lalàna no mamaritra
ny momba ny mpikambana sy ny andraikiny ary ny fomba entiny miasa.

\end{document}